\documentclass[11pt]{article}

    \usepackage[breakable]{tcolorbox}
    \usepackage{parskip} % Stop auto-indenting (to mimic markdown behaviour)
    

    % Basic figure setup, for now with no caption control since it's done
    % automatically by Pandoc (which extracts ![](path) syntax from Markdown).
    \usepackage{graphicx}
    % Maintain compatibility with old templates. Remove in nbconvert 6.0
    \let\Oldincludegraphics\includegraphics
    % Ensure that by default, figures have no caption (until we provide a
    % proper Figure object with a Caption API and a way to capture that
    % in the conversion process - todo).
    \usepackage{caption}
    \DeclareCaptionFormat{nocaption}{}
    \captionsetup{format=nocaption,aboveskip=0pt,belowskip=0pt}

    \usepackage{float}
    \floatplacement{figure}{H} % forces figures to be placed at the correct location
    \usepackage{xcolor} % Allow colors to be defined
    \usepackage{enumerate} % Needed for markdown enumerations to work
    \usepackage{geometry} % Used to adjust the document margins
    \usepackage{amsmath} % Equations
    \usepackage{amssymb} % Equations
    \usepackage{textcomp} % defines textquotesingle
    % Hack from http://tex.stackexchange.com/a/47451/13684:
    \AtBeginDocument{%
        \def\PYZsq{\textquotesingle}% Upright quotes in Pygmentized code
    }
    \usepackage{upquote} % Upright quotes for verbatim code
    \usepackage{eurosym} % defines \euro

    \usepackage{iftex}
    \ifPDFTeX
        \usepackage[T1]{fontenc}
        \IfFileExists{alphabeta.sty}{
              \usepackage{alphabeta}
          }{
              \usepackage[mathletters]{ucs}
              \usepackage[utf8x]{inputenc}
          }
    \else
        \usepackage{fontspec}
        \usepackage{unicode-math}
    \fi

    \usepackage{fancyvrb} % verbatim replacement that allows latex
    \usepackage{grffile} % extends the file name processing of package graphics
                         % to support a larger range
    \makeatletter % fix for old versions of grffile with XeLaTeX
    \@ifpackagelater{grffile}{2019/11/01}
    {
      % Do nothing on new versions
    }
    {
      \def\Gread@@xetex#1{%
        \IfFileExists{"\Gin@base".bb}%
        {\Gread@eps{\Gin@base.bb}}%
        {\Gread@@xetex@aux#1}%
      }
    }
    \makeatother
    \usepackage[Export]{adjustbox} % Used to constrain images to a maximum size
    \adjustboxset{max size={0.9\linewidth}{0.9\paperheight}}

    % The hyperref package gives us a pdf with properly built
    % internal navigation ('pdf bookmarks' for the table of contents,
    % internal cross-reference links, web links for URLs, etc.)
    \usepackage{hyperref}
    % The default LaTeX title has an obnoxious amount of whitespace. By default,
    % titling removes some of it. It also provides customization options.
    \usepackage{titling}
    \usepackage{longtable} % longtable support required by pandoc >1.10
    \usepackage{booktabs}  % table support for pandoc > 1.12.2
    \usepackage{array}     % table support for pandoc >= 2.11.3
    \usepackage{calc}      % table minipage width calculation for pandoc >= 2.11.1
    \usepackage[inline]{enumitem} % IRkernel/repr support (it uses the enumerate* environment)
    \usepackage[normalem]{ulem} % ulem is needed to support strikethroughs (\sout)
                                % normalem makes italics be italics, not underlines
    \usepackage{mathrsfs}
    

    
    % Colors for the hyperref package
    \definecolor{urlcolor}{rgb}{0,.145,.698}
    \definecolor{linkcolor}{rgb}{.71,0.21,0.01}
    \definecolor{citecolor}{rgb}{.12,.54,.11}

    % ANSI colors
    \definecolor{ansi-black}{HTML}{3E424D}
    \definecolor{ansi-black-intense}{HTML}{282C36}
    \definecolor{ansi-red}{HTML}{E75C58}
    \definecolor{ansi-red-intense}{HTML}{B22B31}
    \definecolor{ansi-green}{HTML}{00A250}
    \definecolor{ansi-green-intense}{HTML}{007427}
    \definecolor{ansi-yellow}{HTML}{DDB62B}
    \definecolor{ansi-yellow-intense}{HTML}{B27D12}
    \definecolor{ansi-blue}{HTML}{208FFB}
    \definecolor{ansi-blue-intense}{HTML}{0065CA}
    \definecolor{ansi-magenta}{HTML}{D160C4}
    \definecolor{ansi-magenta-intense}{HTML}{A03196}
    \definecolor{ansi-cyan}{HTML}{60C6C8}
    \definecolor{ansi-cyan-intense}{HTML}{258F8F}
    \definecolor{ansi-white}{HTML}{C5C1B4}
    \definecolor{ansi-white-intense}{HTML}{A1A6B2}
    \definecolor{ansi-default-inverse-fg}{HTML}{FFFFFF}
    \definecolor{ansi-default-inverse-bg}{HTML}{000000}

    % common color for the border for error outputs.
    \definecolor{outerrorbackground}{HTML}{FFDFDF}

    % commands and environments needed by pandoc snippets
    % extracted from the output of `pandoc -s`
    \providecommand{\tightlist}{%
      \setlength{\itemsep}{0pt}\setlength{\parskip}{0pt}}
    \DefineVerbatimEnvironment{Highlighting}{Verbatim}{commandchars=\\\{\}}
    % Add ',fontsize=\small' for more characters per line
    \newenvironment{Shaded}{}{}
    \newcommand{\KeywordTok}[1]{\textcolor[rgb]{0.00,0.44,0.13}{\textbf{{#1}}}}
    \newcommand{\DataTypeTok}[1]{\textcolor[rgb]{0.56,0.13,0.00}{{#1}}}
    \newcommand{\DecValTok}[1]{\textcolor[rgb]{0.25,0.63,0.44}{{#1}}}
    \newcommand{\BaseNTok}[1]{\textcolor[rgb]{0.25,0.63,0.44}{{#1}}}
    \newcommand{\FloatTok}[1]{\textcolor[rgb]{0.25,0.63,0.44}{{#1}}}
    \newcommand{\CharTok}[1]{\textcolor[rgb]{0.25,0.44,0.63}{{#1}}}
    \newcommand{\StringTok}[1]{\textcolor[rgb]{0.25,0.44,0.63}{{#1}}}
    \newcommand{\CommentTok}[1]{\textcolor[rgb]{0.38,0.63,0.69}{\textit{{#1}}}}
    \newcommand{\OtherTok}[1]{\textcolor[rgb]{0.00,0.44,0.13}{{#1}}}
    \newcommand{\AlertTok}[1]{\textcolor[rgb]{1.00,0.00,0.00}{\textbf{{#1}}}}
    \newcommand{\FunctionTok}[1]{\textcolor[rgb]{0.02,0.16,0.49}{{#1}}}
    \newcommand{\RegionMarkerTok}[1]{{#1}}
    \newcommand{\ErrorTok}[1]{\textcolor[rgb]{1.00,0.00,0.00}{\textbf{{#1}}}}
    \newcommand{\NormalTok}[1]{{#1}}

    % Additional commands for more recent versions of Pandoc
    \newcommand{\ConstantTok}[1]{\textcolor[rgb]{0.53,0.00,0.00}{{#1}}}
    \newcommand{\SpecialCharTok}[1]{\textcolor[rgb]{0.25,0.44,0.63}{{#1}}}
    \newcommand{\VerbatimStringTok}[1]{\textcolor[rgb]{0.25,0.44,0.63}{{#1}}}
    \newcommand{\SpecialStringTok}[1]{\textcolor[rgb]{0.73,0.40,0.53}{{#1}}}
    \newcommand{\ImportTok}[1]{{#1}}
    \newcommand{\DocumentationTok}[1]{\textcolor[rgb]{0.73,0.13,0.13}{\textit{{#1}}}}
    \newcommand{\AnnotationTok}[1]{\textcolor[rgb]{0.38,0.63,0.69}{\textbf{\textit{{#1}}}}}
    \newcommand{\CommentVarTok}[1]{\textcolor[rgb]{0.38,0.63,0.69}{\textbf{\textit{{#1}}}}}
    \newcommand{\VariableTok}[1]{\textcolor[rgb]{0.10,0.09,0.49}{{#1}}}
    \newcommand{\ControlFlowTok}[1]{\textcolor[rgb]{0.00,0.44,0.13}{\textbf{{#1}}}}
    \newcommand{\OperatorTok}[1]{\textcolor[rgb]{0.40,0.40,0.40}{{#1}}}
    \newcommand{\BuiltInTok}[1]{{#1}}
    \newcommand{\ExtensionTok}[1]{{#1}}
    \newcommand{\PreprocessorTok}[1]{\textcolor[rgb]{0.74,0.48,0.00}{{#1}}}
    \newcommand{\AttributeTok}[1]{\textcolor[rgb]{0.49,0.56,0.16}{{#1}}}
    \newcommand{\InformationTok}[1]{\textcolor[rgb]{0.38,0.63,0.69}{\textbf{\textit{{#1}}}}}
    \newcommand{\WarningTok}[1]{\textcolor[rgb]{0.38,0.63,0.69}{\textbf{\textit{{#1}}}}}


    % Define a nice break command that doesn't care if a line doesn't already
    % exist.
    \def\br{\hspace*{\fill} \\* }
    % Math Jax compatibility definitions
    \def\gt{>}
    \def\lt{<}
    \let\Oldtex\TeX
    \let\Oldlatex\LaTeX
    \renewcommand{\TeX}{\textrm{\Oldtex}}
    \renewcommand{\LaTeX}{\textrm{\Oldlatex}}
    % Document parameters
    % Document title
    \title{Es. 1 (introduzione a Python, le basi)}
    
    
    
    
    
% Pygments definitions
\makeatletter
\def\PY@reset{\let\PY@it=\relax \let\PY@bf=\relax%
    \let\PY@ul=\relax \let\PY@tc=\relax%
    \let\PY@bc=\relax \let\PY@ff=\relax}
\def\PY@tok#1{\csname PY@tok@#1\endcsname}
\def\PY@toks#1+{\ifx\relax#1\empty\else%
    \PY@tok{#1}\expandafter\PY@toks\fi}
\def\PY@do#1{\PY@bc{\PY@tc{\PY@ul{%
    \PY@it{\PY@bf{\PY@ff{#1}}}}}}}
\def\PY#1#2{\PY@reset\PY@toks#1+\relax+\PY@do{#2}}

\@namedef{PY@tok@w}{\def\PY@tc##1{\textcolor[rgb]{0.73,0.73,0.73}{##1}}}
\@namedef{PY@tok@c}{\let\PY@it=\textit\def\PY@tc##1{\textcolor[rgb]{0.24,0.48,0.48}{##1}}}
\@namedef{PY@tok@cp}{\def\PY@tc##1{\textcolor[rgb]{0.61,0.40,0.00}{##1}}}
\@namedef{PY@tok@k}{\let\PY@bf=\textbf\def\PY@tc##1{\textcolor[rgb]{0.00,0.50,0.00}{##1}}}
\@namedef{PY@tok@kp}{\def\PY@tc##1{\textcolor[rgb]{0.00,0.50,0.00}{##1}}}
\@namedef{PY@tok@kt}{\def\PY@tc##1{\textcolor[rgb]{0.69,0.00,0.25}{##1}}}
\@namedef{PY@tok@o}{\def\PY@tc##1{\textcolor[rgb]{0.40,0.40,0.40}{##1}}}
\@namedef{PY@tok@ow}{\let\PY@bf=\textbf\def\PY@tc##1{\textcolor[rgb]{0.67,0.13,1.00}{##1}}}
\@namedef{PY@tok@nb}{\def\PY@tc##1{\textcolor[rgb]{0.00,0.50,0.00}{##1}}}
\@namedef{PY@tok@nf}{\def\PY@tc##1{\textcolor[rgb]{0.00,0.00,1.00}{##1}}}
\@namedef{PY@tok@nc}{\let\PY@bf=\textbf\def\PY@tc##1{\textcolor[rgb]{0.00,0.00,1.00}{##1}}}
\@namedef{PY@tok@nn}{\let\PY@bf=\textbf\def\PY@tc##1{\textcolor[rgb]{0.00,0.00,1.00}{##1}}}
\@namedef{PY@tok@ne}{\let\PY@bf=\textbf\def\PY@tc##1{\textcolor[rgb]{0.80,0.25,0.22}{##1}}}
\@namedef{PY@tok@nv}{\def\PY@tc##1{\textcolor[rgb]{0.10,0.09,0.49}{##1}}}
\@namedef{PY@tok@no}{\def\PY@tc##1{\textcolor[rgb]{0.53,0.00,0.00}{##1}}}
\@namedef{PY@tok@nl}{\def\PY@tc##1{\textcolor[rgb]{0.46,0.46,0.00}{##1}}}
\@namedef{PY@tok@ni}{\let\PY@bf=\textbf\def\PY@tc##1{\textcolor[rgb]{0.44,0.44,0.44}{##1}}}
\@namedef{PY@tok@na}{\def\PY@tc##1{\textcolor[rgb]{0.41,0.47,0.13}{##1}}}
\@namedef{PY@tok@nt}{\let\PY@bf=\textbf\def\PY@tc##1{\textcolor[rgb]{0.00,0.50,0.00}{##1}}}
\@namedef{PY@tok@nd}{\def\PY@tc##1{\textcolor[rgb]{0.67,0.13,1.00}{##1}}}
\@namedef{PY@tok@s}{\def\PY@tc##1{\textcolor[rgb]{0.73,0.13,0.13}{##1}}}
\@namedef{PY@tok@sd}{\let\PY@it=\textit\def\PY@tc##1{\textcolor[rgb]{0.73,0.13,0.13}{##1}}}
\@namedef{PY@tok@si}{\let\PY@bf=\textbf\def\PY@tc##1{\textcolor[rgb]{0.64,0.35,0.47}{##1}}}
\@namedef{PY@tok@se}{\let\PY@bf=\textbf\def\PY@tc##1{\textcolor[rgb]{0.67,0.36,0.12}{##1}}}
\@namedef{PY@tok@sr}{\def\PY@tc##1{\textcolor[rgb]{0.64,0.35,0.47}{##1}}}
\@namedef{PY@tok@ss}{\def\PY@tc##1{\textcolor[rgb]{0.10,0.09,0.49}{##1}}}
\@namedef{PY@tok@sx}{\def\PY@tc##1{\textcolor[rgb]{0.00,0.50,0.00}{##1}}}
\@namedef{PY@tok@m}{\def\PY@tc##1{\textcolor[rgb]{0.40,0.40,0.40}{##1}}}
\@namedef{PY@tok@gh}{\let\PY@bf=\textbf\def\PY@tc##1{\textcolor[rgb]{0.00,0.00,0.50}{##1}}}
\@namedef{PY@tok@gu}{\let\PY@bf=\textbf\def\PY@tc##1{\textcolor[rgb]{0.50,0.00,0.50}{##1}}}
\@namedef{PY@tok@gd}{\def\PY@tc##1{\textcolor[rgb]{0.63,0.00,0.00}{##1}}}
\@namedef{PY@tok@gi}{\def\PY@tc##1{\textcolor[rgb]{0.00,0.52,0.00}{##1}}}
\@namedef{PY@tok@gr}{\def\PY@tc##1{\textcolor[rgb]{0.89,0.00,0.00}{##1}}}
\@namedef{PY@tok@ge}{\let\PY@it=\textit}
\@namedef{PY@tok@gs}{\let\PY@bf=\textbf}
\@namedef{PY@tok@gp}{\let\PY@bf=\textbf\def\PY@tc##1{\textcolor[rgb]{0.00,0.00,0.50}{##1}}}
\@namedef{PY@tok@go}{\def\PY@tc##1{\textcolor[rgb]{0.44,0.44,0.44}{##1}}}
\@namedef{PY@tok@gt}{\def\PY@tc##1{\textcolor[rgb]{0.00,0.27,0.87}{##1}}}
\@namedef{PY@tok@err}{\def\PY@bc##1{{\setlength{\fboxsep}{\string -\fboxrule}\fcolorbox[rgb]{1.00,0.00,0.00}{1,1,1}{\strut ##1}}}}
\@namedef{PY@tok@kc}{\let\PY@bf=\textbf\def\PY@tc##1{\textcolor[rgb]{0.00,0.50,0.00}{##1}}}
\@namedef{PY@tok@kd}{\let\PY@bf=\textbf\def\PY@tc##1{\textcolor[rgb]{0.00,0.50,0.00}{##1}}}
\@namedef{PY@tok@kn}{\let\PY@bf=\textbf\def\PY@tc##1{\textcolor[rgb]{0.00,0.50,0.00}{##1}}}
\@namedef{PY@tok@kr}{\let\PY@bf=\textbf\def\PY@tc##1{\textcolor[rgb]{0.00,0.50,0.00}{##1}}}
\@namedef{PY@tok@bp}{\def\PY@tc##1{\textcolor[rgb]{0.00,0.50,0.00}{##1}}}
\@namedef{PY@tok@fm}{\def\PY@tc##1{\textcolor[rgb]{0.00,0.00,1.00}{##1}}}
\@namedef{PY@tok@vc}{\def\PY@tc##1{\textcolor[rgb]{0.10,0.09,0.49}{##1}}}
\@namedef{PY@tok@vg}{\def\PY@tc##1{\textcolor[rgb]{0.10,0.09,0.49}{##1}}}
\@namedef{PY@tok@vi}{\def\PY@tc##1{\textcolor[rgb]{0.10,0.09,0.49}{##1}}}
\@namedef{PY@tok@vm}{\def\PY@tc##1{\textcolor[rgb]{0.10,0.09,0.49}{##1}}}
\@namedef{PY@tok@sa}{\def\PY@tc##1{\textcolor[rgb]{0.73,0.13,0.13}{##1}}}
\@namedef{PY@tok@sb}{\def\PY@tc##1{\textcolor[rgb]{0.73,0.13,0.13}{##1}}}
\@namedef{PY@tok@sc}{\def\PY@tc##1{\textcolor[rgb]{0.73,0.13,0.13}{##1}}}
\@namedef{PY@tok@dl}{\def\PY@tc##1{\textcolor[rgb]{0.73,0.13,0.13}{##1}}}
\@namedef{PY@tok@s2}{\def\PY@tc##1{\textcolor[rgb]{0.73,0.13,0.13}{##1}}}
\@namedef{PY@tok@sh}{\def\PY@tc##1{\textcolor[rgb]{0.73,0.13,0.13}{##1}}}
\@namedef{PY@tok@s1}{\def\PY@tc##1{\textcolor[rgb]{0.73,0.13,0.13}{##1}}}
\@namedef{PY@tok@mb}{\def\PY@tc##1{\textcolor[rgb]{0.40,0.40,0.40}{##1}}}
\@namedef{PY@tok@mf}{\def\PY@tc##1{\textcolor[rgb]{0.40,0.40,0.40}{##1}}}
\@namedef{PY@tok@mh}{\def\PY@tc##1{\textcolor[rgb]{0.40,0.40,0.40}{##1}}}
\@namedef{PY@tok@mi}{\def\PY@tc##1{\textcolor[rgb]{0.40,0.40,0.40}{##1}}}
\@namedef{PY@tok@il}{\def\PY@tc##1{\textcolor[rgb]{0.40,0.40,0.40}{##1}}}
\@namedef{PY@tok@mo}{\def\PY@tc##1{\textcolor[rgb]{0.40,0.40,0.40}{##1}}}
\@namedef{PY@tok@ch}{\let\PY@it=\textit\def\PY@tc##1{\textcolor[rgb]{0.24,0.48,0.48}{##1}}}
\@namedef{PY@tok@cm}{\let\PY@it=\textit\def\PY@tc##1{\textcolor[rgb]{0.24,0.48,0.48}{##1}}}
\@namedef{PY@tok@cpf}{\let\PY@it=\textit\def\PY@tc##1{\textcolor[rgb]{0.24,0.48,0.48}{##1}}}
\@namedef{PY@tok@c1}{\let\PY@it=\textit\def\PY@tc##1{\textcolor[rgb]{0.24,0.48,0.48}{##1}}}
\@namedef{PY@tok@cs}{\let\PY@it=\textit\def\PY@tc##1{\textcolor[rgb]{0.24,0.48,0.48}{##1}}}

\def\PYZbs{\char`\\}
\def\PYZus{\char`\_}
\def\PYZob{\char`\{}
\def\PYZcb{\char`\}}
\def\PYZca{\char`\^}
\def\PYZam{\char`\&}
\def\PYZlt{\char`\<}
\def\PYZgt{\char`\>}
\def\PYZsh{\char`\#}
\def\PYZpc{\char`\%}
\def\PYZdl{\char`\$}
\def\PYZhy{\char`\-}
\def\PYZsq{\char`\'}
\def\PYZdq{\char`\"}
\def\PYZti{\char`\~}
% for compatibility with earlier versions
\def\PYZat{@}
\def\PYZlb{[}
\def\PYZrb{]}
\makeatother


    % For linebreaks inside Verbatim environment from package fancyvrb.
    \makeatletter
        \newbox\Wrappedcontinuationbox
        \newbox\Wrappedvisiblespacebox
        \newcommand*\Wrappedvisiblespace {\textcolor{red}{\textvisiblespace}}
        \newcommand*\Wrappedcontinuationsymbol {\textcolor{red}{\llap{\tiny$\m@th\hookrightarrow$}}}
        \newcommand*\Wrappedcontinuationindent {3ex }
        \newcommand*\Wrappedafterbreak {\kern\Wrappedcontinuationindent\copy\Wrappedcontinuationbox}
        % Take advantage of the already applied Pygments mark-up to insert
        % potential linebreaks for TeX processing.
        %        {, <, #, %, $, ' and ": go to next line.
        %        _, }, ^, &, >, - and ~: stay at end of broken line.
        % Use of \textquotesingle for straight quote.
        \newcommand*\Wrappedbreaksatspecials {%
            \def\PYGZus{\discretionary{\char`\_}{\Wrappedafterbreak}{\char`\_}}%
            \def\PYGZob{\discretionary{}{\Wrappedafterbreak\char`\{}{\char`\{}}%
            \def\PYGZcb{\discretionary{\char`\}}{\Wrappedafterbreak}{\char`\}}}%
            \def\PYGZca{\discretionary{\char`\^}{\Wrappedafterbreak}{\char`\^}}%
            \def\PYGZam{\discretionary{\char`\&}{\Wrappedafterbreak}{\char`\&}}%
            \def\PYGZlt{\discretionary{}{\Wrappedafterbreak\char`\<}{\char`\<}}%
            \def\PYGZgt{\discretionary{\char`\>}{\Wrappedafterbreak}{\char`\>}}%
            \def\PYGZsh{\discretionary{}{\Wrappedafterbreak\char`\#}{\char`\#}}%
            \def\PYGZpc{\discretionary{}{\Wrappedafterbreak\char`\%}{\char`\%}}%
            \def\PYGZdl{\discretionary{}{\Wrappedafterbreak\char`\$}{\char`\$}}%
            \def\PYGZhy{\discretionary{\char`\-}{\Wrappedafterbreak}{\char`\-}}%
            \def\PYGZsq{\discretionary{}{\Wrappedafterbreak\textquotesingle}{\textquotesingle}}%
            \def\PYGZdq{\discretionary{}{\Wrappedafterbreak\char`\"}{\char`\"}}%
            \def\PYGZti{\discretionary{\char`\~}{\Wrappedafterbreak}{\char`\~}}%
        }
        % Some characters . , ; ? ! / are not pygmentized.
        % This macro makes them "active" and they will insert potential linebreaks
        \newcommand*\Wrappedbreaksatpunct {%
            \lccode`\~`\.\lowercase{\def~}{\discretionary{\hbox{\char`\.}}{\Wrappedafterbreak}{\hbox{\char`\.}}}%
            \lccode`\~`\,\lowercase{\def~}{\discretionary{\hbox{\char`\,}}{\Wrappedafterbreak}{\hbox{\char`\,}}}%
            \lccode`\~`\;\lowercase{\def~}{\discretionary{\hbox{\char`\;}}{\Wrappedafterbreak}{\hbox{\char`\;}}}%
            \lccode`\~`\:\lowercase{\def~}{\discretionary{\hbox{\char`\:}}{\Wrappedafterbreak}{\hbox{\char`\:}}}%
            \lccode`\~`\?\lowercase{\def~}{\discretionary{\hbox{\char`\?}}{\Wrappedafterbreak}{\hbox{\char`\?}}}%
            \lccode`\~`\!\lowercase{\def~}{\discretionary{\hbox{\char`\!}}{\Wrappedafterbreak}{\hbox{\char`\!}}}%
            \lccode`\~`\/\lowercase{\def~}{\discretionary{\hbox{\char`\/}}{\Wrappedafterbreak}{\hbox{\char`\/}}}%
            \catcode`\.\active
            \catcode`\,\active
            \catcode`\;\active
            \catcode`\:\active
            \catcode`\?\active
            \catcode`\!\active
            \catcode`\/\active
            \lccode`\~`\~
        }
    \makeatother

    \let\OriginalVerbatim=\Verbatim
    \makeatletter
    \renewcommand{\Verbatim}[1][1]{%
        %\parskip\z@skip
        \sbox\Wrappedcontinuationbox {\Wrappedcontinuationsymbol}%
        \sbox\Wrappedvisiblespacebox {\FV@SetupFont\Wrappedvisiblespace}%
        \def\FancyVerbFormatLine ##1{\hsize\linewidth
            \vtop{\raggedright\hyphenpenalty\z@\exhyphenpenalty\z@
                \doublehyphendemerits\z@\finalhyphendemerits\z@
                \strut ##1\strut}%
        }%
        % If the linebreak is at a space, the latter will be displayed as visible
        % space at end of first line, and a continuation symbol starts next line.
        % Stretch/shrink are however usually zero for typewriter font.
        \def\FV@Space {%
            \nobreak\hskip\z@ plus\fontdimen3\font minus\fontdimen4\font
            \discretionary{\copy\Wrappedvisiblespacebox}{\Wrappedafterbreak}
            {\kern\fontdimen2\font}%
        }%

        % Allow breaks at special characters using \PYG... macros.
        \Wrappedbreaksatspecials
        % Breaks at punctuation characters . , ; ? ! and / need catcode=\active
        \OriginalVerbatim[#1,codes*=\Wrappedbreaksatpunct]%
    }
    \makeatother

    % Exact colors from NB
    \definecolor{incolor}{HTML}{303F9F}
    \definecolor{outcolor}{HTML}{D84315}
    \definecolor{cellborder}{HTML}{CFCFCF}
    \definecolor{cellbackground}{HTML}{F7F7F7}

    % prompt
    \makeatletter
    \newcommand{\boxspacing}{\kern\kvtcb@left@rule\kern\kvtcb@boxsep}
    \makeatother
    \newcommand{\prompt}[4]{
        {\ttfamily\llap{{\color{#2}[#3]:\hspace{3pt}#4}}\vspace{-\baselineskip}}
    }
    

    
    % Prevent overflowing lines due to hard-to-break entities
    \sloppy
    % Setup hyperref package
    \hypersetup{
      breaklinks=true,  % so long urls are correctly broken across lines
      colorlinks=true,
      urlcolor=urlcolor,
      linkcolor=linkcolor,
      citecolor=citecolor,
      }
    % Slightly bigger margins than the latex defaults
    
    \geometry{verbose,tmargin=1in,bmargin=1in,lmargin=1in,rmargin=1in}
    
    

\begin{document}
    
    \maketitle
    
    

    
    \section{LE BASI DI PYTHON - IL MIO PRIMO
PROGRAMMA}\label{le-basi-di-python---il-mio-primo-programma}

    Questo breve codice utilizza il comando print per mostrare il messaggio
``Hello, World!'' a schermo. In Python, il print è il comando utilizzato
per visualizzare testo o risultati durante l'esecuzione di un programma.
Nel caso specifico, il programma stampa semplicemente un saluto di base.
L'Hello World è il programma esempio ``base'' di un qualsiasi linguaggio
di programmazione; infatti, come in questo caso, è il primo codice che
si apprende.

    \begin{tcolorbox}[breakable, size=fbox, boxrule=1pt, pad at break*=1mm,colback=cellbackground, colframe=cellborder]
\prompt{In}{incolor}{2}{\boxspacing}
\begin{Verbatim}[commandchars=\\\{\}]
\PY{c+c1}{\PYZsh{}Comprendere il comando print}
\PY{n+nb}{print}\PY{p}{(}\PY{l+s+s2}{\PYZdq{}}\PY{l+s+s2}{Hello, World!}\PY{l+s+s2}{\PYZdq{}}\PY{p}{)}\PY{c+c1}{\PYZsh{}il comando print stampa un testo indicato nelle virgolette, quello spazio è inteso come una stringa di testo}
\end{Verbatim}
\end{tcolorbox}

    \begin{Verbatim}[commandchars=\\\{\}]
Hello, World!
    \end{Verbatim}

    Questo codice utilizza il comando print per mostrare il contenuto della
variabile nome. Il programma assegna la stringa ``Matteo'' alla
variabile nome, la stringa non è altro che una frase di testo normale in
linguaggio ``umano'' come è per l'appunto la parola ``Matteo'' che è una
parola del linguaggio ``umano'' che viene assegnata come dato ad una
variabile ed è una stringa. Successivamente, il programma utilizza il
comando print per visualizzare il messaggio ``Il nome esempio è:''
seguito dal contenuto della variabile nome. Questo significa che verrà
stampato a schermo ``Il nome esempio è: Matteo'', ovviamente la parola
``Matteo'' è a capo perchè i due print sono in due righe diverse nel
codice e questo fa sì che il risultato finale venga ``printato'' in due
righe separate. In sintesi, il comando print può essere utilizzato anche
per visualizzare il contenuto di variabili, come il nome nella variabile
``nome'', o di qualsiasi altro dato.

    \begin{tcolorbox}[breakable, size=fbox, boxrule=1pt, pad at break*=1mm,colback=cellbackground, colframe=cellborder]
\prompt{In}{incolor}{1}{\boxspacing}
\begin{Verbatim}[commandchars=\\\{\}]
\PY{c+c1}{\PYZsh{}Comprendere il comando print e comprendere come definire una variabile con una stringa}
\PY{n}{nome}\PY{o}{=}\PY{l+s+s2}{\PYZdq{}}\PY{l+s+s2}{Matteo}\PY{l+s+s2}{\PYZdq{}}
\PY{n+nb}{print}\PY{p}{(}\PY{l+s+s2}{\PYZdq{}}\PY{l+s+s2}{Il nome esempio è:}\PY{l+s+s2}{\PYZdq{}}\PY{p}{)}
\PY{n+nb}{print}\PY{p}{(}\PY{n}{nome}\PY{p}{)}
\end{Verbatim}
\end{tcolorbox}

    \begin{Verbatim}[commandchars=\\\{\}]
Il nome esempio è:
Matteo
    \end{Verbatim}

    \section{LE VARIABILI E L'INPUT
UTENTE}\label{le-variabili-e-linput-utente}

    Questo codice utilizza il comando input per acquisire dati dall'utente.
Ecco una breve spiegazione di come funziona: Il programma per prima cosa
chiede all'utente di inserire il proprio nome utilizzando il comando
input, l'input viene quindi memorizzato come un dato nella variabile
nome. Successivamente, viene richiesto all'utente di inserire il cognome
utilizzando l'altro input, quest'ultimo a quel punto viene sempre
memorizzato come un dato nella variabile cognome. Infine, il programma
stampa un messaggio di benvenuto utilizzando il comando
print(``Benvenuto su Jupyter Notebook'', nome, cognome), dove nome e
cognome sono i dati inseriti dall'utente poichè sono le variabili e
quindi il programma stamperà direttamente i loro valori, cioè i dati che
aveva immesso prima l'utente nell'input. Questo codice è un esempio
semplice ma utile di come interagire con l'utente attraverso l'input e
di come salvare i dati corretti nelle variabili.

    \begin{tcolorbox}[breakable, size=fbox, boxrule=1pt, pad at break*=1mm,colback=cellbackground, colframe=cellborder]
\prompt{In}{incolor}{1}{\boxspacing}
\begin{Verbatim}[commandchars=\\\{\}]
\PY{c+c1}{\PYZsh{}Comprendere il comando input}
\PY{n}{nome}\PY{o}{=}\PY{n+nb}{input}\PY{p}{(}\PY{l+s+s2}{\PYZdq{}}\PY{l+s+s2}{Inserisci il tuo nome: }\PY{l+s+s2}{\PYZdq{}}\PY{p}{)}\PY{c+c1}{\PYZsh{}l\PYZsq{}input \PYZdq{}prende\PYZdq{} un dato dall\PYZsq{}utente e lo salva temporaneamente nella RAM (Random Access Memory)}
\PY{n}{cognome}\PY{o}{=}\PY{n+nb}{input}\PY{p}{(}\PY{l+s+s2}{\PYZdq{}}\PY{l+s+s2}{Inserisci il tuo cognome: }\PY{l+s+s2}{\PYZdq{}}\PY{p}{)}\PY{c+c1}{\PYZsh{}le variabili si possono indicare come si vogliono però è meglio, per se stessi e per gli altri, che visualizzeranno il codice usare dei nomi sensati per ogni occasione}
\PY{n+nb}{print}\PY{p}{(}\PY{l+s+s2}{\PYZdq{}}\PY{l+s+s2}{Benvenuto su Jupyter Notebook}\PY{l+s+s2}{\PYZdq{}}\PY{p}{,}\PY{n}{nome}\PY{p}{,}\PY{n}{cognome}\PY{p}{)}\PY{c+c1}{\PYZsh{}dentro il print ci sono due input dettati dall\PYZsq{}utente che specificano i dati}
\end{Verbatim}
\end{tcolorbox}

    \begin{Verbatim}[commandchars=\\\{\}]
Inserisci il tuo nome: Matteo
Inserisci il tuo cognome: Magrino
Benvenuto su Jupyter Notebook Matteo Magrino
    \end{Verbatim}

    Il codice qui sotto chiede all'utente di inserire il nome completo della
via e l'indirizzo di casa e successivamente stampa il messaggio
confermando la via inserita. Questo codice è identico strutturalmente a
quello di sopra perchè chiede sempre un informazione all'utente via
input, solo che questo è un esempio diverso

CURIOSITÀ: la via 20 W 34th St., New York, NY 10001, Stati Uniti è
quella dove risiede una delle due facciate laterali lunghe del
famossisimo grattacielo ``Empire State Building''. Link alla pagina di
Google Maps: https://bit.ly/googlemapsempirestatebuilding

    \begin{tcolorbox}[breakable, size=fbox, boxrule=1pt, pad at break*=1mm,colback=cellbackground, colframe=cellborder]
\prompt{In}{incolor}{1}{\boxspacing}
\begin{Verbatim}[commandchars=\\\{\}]
\PY{n}{indirizzoviadicasa}\PY{o}{=}\PY{n+nb}{input}\PY{p}{(}\PY{l+s+s2}{\PYZdq{}}\PY{l+s+s2}{Inserisci il nome completo della via, nonchè l}\PY{l+s+s2}{\PYZsq{}}\PY{l+s+s2}{indirizzo, di casa tua: }\PY{l+s+s2}{\PYZdq{}}\PY{p}{)}
\PY{n+nb}{print}\PY{p}{(}\PY{l+s+s2}{\PYZdq{}}\PY{l+s+s2}{Hai inserito la via:}\PY{l+s+s2}{\PYZdq{}}\PY{p}{,}\PY{n}{indirizzoviadicasa}\PY{p}{)}
\end{Verbatim}
\end{tcolorbox}

    \begin{Verbatim}[commandchars=\\\{\}]
Inserisci il nome completo della via, nonchè l'indirizzo, di casa tua: 20 W 34th
St., New York, NY 10001, Stati Uniti
Hai inserito la via: 20 W 34th St., New York, NY 10001, Stati Uniti
    \end{Verbatim}

    \section{CONCATENAZIONI E MOLTIPLICAZIONI CON LE STRINGHE DI
TESTO}\label{concatenazioni-e-moltiplicazioni-con-le-stringhe-di-testo}

    Il codice ``Matteo'' + ``Magrino'' esegue la concatenazione di due
stringhe, ``Matteo'' e ``Magrino'', producendo così una nuova stringa
risultante dalla fusione delle due. La concatenazione delle stringhe è
un'operazione molto comune in programmazione, utile per unire più
stringhe in una sola e fare così un output unico.

    \begin{tcolorbox}[breakable, size=fbox, boxrule=1pt, pad at break*=1mm,colback=cellbackground, colframe=cellborder]
\prompt{In}{incolor}{1}{\boxspacing}
\begin{Verbatim}[commandchars=\\\{\}]
\PY{l+s+s2}{\PYZdq{}}\PY{l+s+s2}{Matteo}\PY{l+s+s2}{\PYZdq{}}\PY{o}{+}\PY{l+s+s2}{\PYZdq{}}\PY{l+s+s2}{Magrino}\PY{l+s+s2}{\PYZdq{}}
\end{Verbatim}
\end{tcolorbox}

            \begin{tcolorbox}[breakable, size=fbox, boxrule=.5pt, pad at break*=1mm, opacityfill=0]
\prompt{Out}{outcolor}{1}{\boxspacing}
\begin{Verbatim}[commandchars=\\\{\}]
'MatteoMagrino'
\end{Verbatim}
\end{tcolorbox}
        
    Il codice qui sotto è identico a quello precedente ma non è presente il
simbolo ``+'', infatti la concatenazione delle stringhe si può creare
anche senza il simbolo +, questo perchè in Python se le due stringhe si
trovano una accanto all'altra senza un operatore tra di esse, vengono
automaticamente concatenate. Questo è noto come ``concatenazione
implicita delle stringhe''.

    \begin{tcolorbox}[breakable, size=fbox, boxrule=1pt, pad at break*=1mm,colback=cellbackground, colframe=cellborder]
\prompt{In}{incolor}{7}{\boxspacing}
\begin{Verbatim}[commandchars=\\\{\}]
\PY{l+s+s2}{\PYZdq{}}\PY{l+s+s2}{Matteo}\PY{l+s+s2}{\PYZdq{}}\PY{l+s+s2}{\PYZdq{}}\PY{l+s+s2}{Magrino}\PY{l+s+s2}{\PYZdq{}}
\end{Verbatim}
\end{tcolorbox}

            \begin{tcolorbox}[breakable, size=fbox, boxrule=.5pt, pad at break*=1mm, opacityfill=0]
\prompt{Out}{outcolor}{7}{\boxspacing}
\begin{Verbatim}[commandchars=\\\{\}]
'MatteoMagrino'
\end{Verbatim}
\end{tcolorbox}
        
    In Python, moltiplicare una stringa per un numero intero produce una
nuova stringa che consiste nella stringa originale ripetuta per un certo
numero di volte. Quindi, l'espressione ``Matteo'' * 30 restituirà come
output una stringa contenente il valore della stringa ``Matteo''
ripetuto 30 volte. Questo approccio è utile quando si desidera creare
rapidamente una stringa ripetuta più volte senza doverla scrivere
esplicitamente più volte nel codice, rendendo il codice più conciso,
efficiente e veloce da scrivere. Ad esempio, se volessimo creare una
stringa che contenga il nome ``Matteo'' ripetuto 30 volte, sarebbe più
efficiente e conciso scrivere ``Matteo'' * 30 piuttosto che scrivere
manualmente ``Matteo'' 30 volte, inoltre il codice occuperebbe più
memoria. Questo approccio è particolarmente utile quando si desidera
creare stringhe di grandi dimensioni o ripetere un certo pattern
(ripetività) all'interno di una stringa.

    \begin{tcolorbox}[breakable, size=fbox, boxrule=1pt, pad at break*=1mm,colback=cellbackground, colframe=cellborder]
\prompt{In}{incolor}{13}{\boxspacing}
\begin{Verbatim}[commandchars=\\\{\}]
\PY{l+s+s2}{\PYZdq{}}\PY{l+s+s2}{Matteo}\PY{l+s+s2}{\PYZdq{}}\PY{o}{*}\PY{l+m+mi}{30}
\end{Verbatim}
\end{tcolorbox}

            \begin{tcolorbox}[breakable, size=fbox, boxrule=.5pt, pad at break*=1mm, opacityfill=0]
\prompt{Out}{outcolor}{13}{\boxspacing}
\begin{Verbatim}[commandchars=\\\{\}]
'MatteoMatteoMatteoMatteoMatteoMatteoMatteoMatteoMatteoMatteoMatteoMatteoMatteoM
atteoMatteoMatteoMatteoMatteoMatteoMatteoMatteoMatteoMatteoMatteoMatteoMatteoMat
teoMatteoMatteoMatteo'
\end{Verbatim}
\end{tcolorbox}
        
    Questo codice è un riassunto di quelli precedenti riguardanti le
stringhe di testo. L'espressione ``Matteo'' * 20 + ``Magrino'' * 50
combina due operazioni: la moltiplicazione di una stringa per un numero
intero e la concatenazione di stringhe. Nel dettaglio, ``Matteo'' * 20
crea una nuova stringa in cui la parola ``Matteo'' viene ripetuta per 20
volte, mentre ``Magrino'' * 50 produce una stringa in cui la parola
``Magrino'' viene ripetuto per 50 volte. Infine, entrambe le stringhe
risultanti vengono concatenate insieme utilizzando l'operatore +,
producendo come output finale una nuova stringa che inizia con
``Matteo'' ripetuto 20 volte seguito poi da ``Magrino'' ripetuto 50
volte. In sintesi, l'output di questa espressione sarà una lunga stringa
che contiene la ripetizione di ``Matteo'' per 20 volte seguita dalla
ripetizione di ``Magrino'' per 50 volte. Un output a dir poco
particolare\ldots{}

    \begin{tcolorbox}[breakable, size=fbox, boxrule=1pt, pad at break*=1mm,colback=cellbackground, colframe=cellborder]
\prompt{In}{incolor}{17}{\boxspacing}
\begin{Verbatim}[commandchars=\\\{\}]
\PY{l+s+s2}{\PYZdq{}}\PY{l+s+s2}{Matteo}\PY{l+s+s2}{\PYZdq{}}\PY{o}{*}\PY{l+m+mi}{20}\PY{o}{+}\PY{l+s+s2}{\PYZdq{}}\PY{l+s+s2}{Magrino}\PY{l+s+s2}{\PYZdq{}}\PY{o}{*}\PY{l+m+mi}{50}
\end{Verbatim}
\end{tcolorbox}

            \begin{tcolorbox}[breakable, size=fbox, boxrule=.5pt, pad at break*=1mm, opacityfill=0]
\prompt{Out}{outcolor}{17}{\boxspacing}
\begin{Verbatim}[commandchars=\\\{\}]
'MatteoMatteoMatteoMatteoMatteoMatteoMatteoMatteoMatteoMatteoMatteoMatteoMatteoM
atteoMatteoMatteoMatteoMatteoMatteoMatteoMagrinoMagrinoMagrinoMagrinoMagrinoMagr
inoMagrinoMagrinoMagrinoMagrinoMagrinoMagrinoMagrinoMagrinoMagrinoMagrinoMagrino
MagrinoMagrinoMagrinoMagrinoMagrinoMagrinoMagrinoMagrinoMagrinoMagrinoMagrinoMag
rinoMagrinoMagrinoMagrinoMagrinoMagrinoMagrinoMagrinoMagrinoMagrinoMagrinoMagrin
oMagrinoMagrinoMagrinoMagrinoMagrinoMagrinoMagrinoMagrinoMagrinoMagrino'
\end{Verbatim}
\end{tcolorbox}
        
    \section{LA CONCATENAZIONE DI
VARIABILI}\label{la-concatenazione-di-variabili}

    In questo codice viene eseguita la concatenazione di variabili per
creare una stringa più complessa e con tutte le informazioni necessarie
all'utente. La concatenazione di variabili consiste nell'unire più
variabili, in questo caso definite come stringhe di testo, per formare
un'unica nuova stringa finale. Nel caso specifico, le variabili
``nome'', ``cognome'' ed ``età'' contengono rispettivamente il nome, il
cognome e l'età di una persona esempio. Queste variabili vengono
concatenate insieme utilizzando l'operatore di concatenazione ``+'' e
del testo aggiuntivo per formare un messaggio più completo e più facile
alla lettura da parte dell'utente.

    \begin{tcolorbox}[breakable, size=fbox, boxrule=1pt, pad at break*=1mm,colback=cellbackground, colframe=cellborder]
\prompt{In}{incolor}{24}{\boxspacing}
\begin{Verbatim}[commandchars=\\\{\}]
\PY{n}{nome}\PY{o}{=}\PY{l+s+s2}{\PYZdq{}}\PY{l+s+s2}{Matteo}\PY{l+s+s2}{\PYZdq{}}
\PY{n}{cognome}\PY{o}{=}\PY{l+s+s2}{\PYZdq{}}\PY{l+s+s2}{Magrino}\PY{l+s+s2}{\PYZdq{}}
\PY{n}{età}\PY{o}{=}\PY{l+s+s2}{\PYZdq{}}\PY{l+s+s2}{15}\PY{l+s+s2}{\PYZdq{}}
\PY{n}{messaggiodastampare}\PY{o}{=}\PY{l+s+s2}{\PYZdq{}}\PY{l+s+s2}{Ciao a tutti mi chiamo }\PY{l+s+s2}{\PYZdq{}}\PY{o}{+}\PY{n}{nome}\PY{o}{+}\PY{l+s+s2}{\PYZdq{}}\PY{l+s+s2}{ }\PY{l+s+s2}{\PYZdq{}}\PY{o}{+}\PY{n}{cognome}\PY{o}{+}\PY{l+s+s2}{\PYZdq{}}\PY{l+s+s2}{, ho }\PY{l+s+s2}{\PYZdq{}}\PY{o}{+}\PY{n}{età}\PY{o}{+}\PY{l+s+s2}{\PYZdq{}}\PY{l+s+s2}{ anni e sono un programmatore in erba di Python e altri linguaggi di programmazione}\PY{l+s+s2}{\PYZdq{}}
\PY{n+nb}{print}\PY{p}{(}\PY{n}{messaggiodastampare}\PY{p}{)}
\end{Verbatim}
\end{tcolorbox}

    \begin{Verbatim}[commandchars=\\\{\}]
Ciao a tutti mi chiamo Matteo Magrino, ho 15 anni e sono un programmatore in
erba di Python e altri linguaggi di programmazione
    \end{Verbatim}

    \section{IL CICLO FOR}\label{il-ciclo-for}

    Il programma sottostante acquisisce il nome e il cognome dell'utente
come input, così come il numero di volte che l'utente desidera ripetere
un messaggio gentile. Utilizza poi un ciclo for per iterare attraverso
il numero di volte specificato dall'utente, gestito dal contatore ``i''
(che sta per iteratore ma si può chiamare come si vuole). In ogni
iterazione del ciclo, il programma stampa un messaggio gentile che
include il nome e il cognome inseriti dall'utente. La parte chiave del
programma è ovviamente il ciclo for, che utilizza un contatore o
iteratore (i due nomi sono sinonimi) ``i'' per iterare attraverso un
numero di volte pari a quello specificato dalla variabile
numeroripetizionimessaggio. In ogni iterazione, il programma stampa un
messaggio gentile che include il nome e il cognome inseriti dall'utente.
Il contatore ``i'' è un vero e proprio iteratore, che tiene traccia
dello stato attuale del ciclo e consente di eseguire azioni specifiche
per ciascuna iterazione. La parte del range(numeroripetizionimessaggio)
fornisce una sequenza di numeri da 0 a numeroripetizionimessaggio-1,
definendo così il numero di iterazioni.

    \begin{tcolorbox}[breakable, size=fbox, boxrule=1pt, pad at break*=1mm,colback=cellbackground, colframe=cellborder]
\prompt{In}{incolor}{1}{\boxspacing}
\begin{Verbatim}[commandchars=\\\{\}]
\PY{c+c1}{\PYZsh{}Comprendere il ciclo for}
\PY{n}{nome}\PY{o}{=}\PY{n+nb}{input}\PY{p}{(}\PY{l+s+s2}{\PYZdq{}}\PY{l+s+s2}{Inserisci il tuo nome: }\PY{l+s+s2}{\PYZdq{}}\PY{p}{)}
\PY{n}{cognome}\PY{o}{=}\PY{n+nb}{input}\PY{p}{(}\PY{l+s+s2}{\PYZdq{}}\PY{l+s+s2}{Inserisci il tuo cognome: }\PY{l+s+s2}{\PYZdq{}}\PY{p}{)}
\PY{n}{numeroripetizionimessaggio}\PY{o}{=}\PY{n+nb}{int}\PY{p}{(}\PY{n+nb}{input}\PY{p}{(}\PY{l+s+s2}{\PYZdq{}}\PY{l+s+s2}{Inserisci il numero di volte che desideri che un messaggio gentile si ripeta: }\PY{l+s+s2}{\PYZdq{}}\PY{p}{)}\PY{p}{)}\PY{c+c1}{\PYZsh{}int è l\PYZsq{}acronimo di integer, cioè di intero e vuole dire che la variabile in questo caso è di tipo intero e può \PYZdq{}leggere\PYZdq{} solo numeri interi (cioè tutti quelli senza virgola) o altro}
\PY{c+c1}{\PYZsh{}For è un contatore o iteratore perchè tiene traccia dello stato attuale del ciclo (per spiegazione più dettagliata su come funziona leggere il riquadro sopra)}
\PY{k}{for} \PY{n}{i} \PY{o+ow}{in} \PY{n+nb}{range}\PY{p}{(}\PY{n}{numeroripetizionimessaggio}\PY{p}{)}\PY{p}{:}\PY{c+c1}{\PYZsh{}il numero di volte che viene ripetuto questo messaggio dipende dall\PYZsq{}input messo precedentemente dall\PYZsq{}utente}
    \PY{n+nb}{print}\PY{p}{(}\PY{l+s+s2}{\PYZdq{}}\PY{l+s+s2}{Che bel nome e cognome che hai}\PY{l+s+s2}{\PYZdq{}}\PY{p}{,}\PY{n}{nome}\PY{p}{,}\PY{n}{cognome}\PY{p}{,}\PY{l+s+s2}{\PYZdq{}}\PY{l+s+s2}{è davvero molto bello}\PY{l+s+s2}{\PYZdq{}}\PY{p}{)}
\end{Verbatim}
\end{tcolorbox}

    \begin{Verbatim}[commandchars=\\\{\}]
Inserisci il tuo nome: Matteo
Inserisci il tuo cognome: Magrino
Inserisci il numero di volte che desideri che un messaggio gentile si ripeta: 12
Che bel nome e cognome che hai Matteo Magrino è davvero molto bello
Che bel nome e cognome che hai Matteo Magrino è davvero molto bello
Che bel nome e cognome che hai Matteo Magrino è davvero molto bello
Che bel nome e cognome che hai Matteo Magrino è davvero molto bello
Che bel nome e cognome che hai Matteo Magrino è davvero molto bello
Che bel nome e cognome che hai Matteo Magrino è davvero molto bello
Che bel nome e cognome che hai Matteo Magrino è davvero molto bello
Che bel nome e cognome che hai Matteo Magrino è davvero molto bello
Che bel nome e cognome che hai Matteo Magrino è davvero molto bello
Che bel nome e cognome che hai Matteo Magrino è davvero molto bello
Che bel nome e cognome che hai Matteo Magrino è davvero molto bello
Che bel nome e cognome che hai Matteo Magrino è davvero molto bello
    \end{Verbatim}

    \section{ESEMPI DI CALCOLATRICE}\label{esempi-di-calcolatrice}

    Il codice inizia con un messaggio di benvenuto, invitando l'utente a
utilizzare la calcolatrice per effettuare addizioni. Successivamente,
attraverso il comando input, si richiede all'utente di inserire due
numeri, che vengono poi convertiti in formato float per consentire la
lettura da parte del programma e l'immisione di numeri decimali da parte
dell'utente. L'operazione di addizione tra i due numeri viene eseguita
utilizzando l'operatore + (più), il risultato viene memorizzato nella
variabile addizione, e infine, il risultato viene visualizzato con un
messaggio appropriato. \textbf{N.B.: la virgola nei numeri decimali in
Python quando vengono immessi come input, come nella maggior parte dei
linguaggi, DEVE sempre essere inserita con un punto e non con una
virgola questo perchè per convenzione si è scelto di assegnare il punto
come separatore decimale}.

    \begin{tcolorbox}[breakable, size=fbox, boxrule=1pt, pad at break*=1mm,colback=cellbackground, colframe=cellborder]
\prompt{In}{incolor}{3}{\boxspacing}
\begin{Verbatim}[commandchars=\\\{\}]
\PY{c+c1}{\PYZsh{}Comprendere come eseguire le addizioni}
\PY{n+nb}{print}\PY{p}{(}\PY{l+s+s2}{\PYZdq{}}\PY{l+s+s2}{Ciao, attraverso questa calcolatrice puoi eseguire calcoli con le addizioni!}\PY{l+s+s2}{\PYZdq{}}\PY{p}{)}
\PY{n}{numero1}\PY{o}{=}\PY{n+nb}{float}\PY{p}{(}\PY{n+nb}{input}\PY{p}{(}\PY{l+s+s2}{\PYZdq{}}\PY{l+s+s2}{Inserisci il primo numero: }\PY{l+s+s2}{\PYZdq{}}\PY{p}{)}\PY{p}{)}\PY{c+c1}{\PYZsh{}float indica che la variabile può leggere i numeri reali, cioè tutti quei numeri inclusi quelli con la virgola o senza}
\PY{n}{numero2}\PY{o}{=}\PY{n+nb}{float}\PY{p}{(}\PY{n+nb}{input}\PY{p}{(}\PY{l+s+s2}{\PYZdq{}}\PY{l+s+s2}{Inserisci il secondo numero: }\PY{l+s+s2}{\PYZdq{}}\PY{p}{)}\PY{p}{)}
\PY{n}{addizione}\PY{o}{=}\PY{n}{numero1}\PY{o}{+}\PY{n}{numero2}
\PY{n+nb}{print}\PY{p}{(}\PY{l+s+s2}{\PYZdq{}}\PY{l+s+s2}{La somma dei due numeri è pari a:}\PY{l+s+s2}{\PYZdq{}}\PY{p}{,}\PY{n+nb}{float}\PY{p}{(}\PY{n}{addizione}\PY{p}{)}\PY{p}{)}
\end{Verbatim}
\end{tcolorbox}

    \begin{Verbatim}[commandchars=\\\{\}]
Ciao, attraverso questa calcolatrice puoi eseguire calcoli con le addizioni!
Inserisci il primo numero: 5.7
Inserisci il secondo numero: 3.2
La somma dei due numeri è pari a: 8.9
    \end{Verbatim}

    Il codice qui sotto svolge l'operazione di sottrazione tra due numeri
(sempre float) inseriti dall'utente. Funziona allo stesso modo ed
infatti ha una identica struttura tranne per il fatto che c'è il simbolo
- (meno) che fa eseguire corretamente l'operazione in questione.

    \begin{tcolorbox}[breakable, size=fbox, boxrule=1pt, pad at break*=1mm,colback=cellbackground, colframe=cellborder]
\prompt{In}{incolor}{2}{\boxspacing}
\begin{Verbatim}[commandchars=\\\{\}]
\PY{c+c1}{\PYZsh{}Comprendere come eseguire le sottrazioni}
\PY{n+nb}{print}\PY{p}{(}\PY{l+s+s2}{\PYZdq{}}\PY{l+s+s2}{Ciao, attraverso questa calcolatrice puoi eseguire calcoli con le sottrazioni!}\PY{l+s+s2}{\PYZdq{}}\PY{p}{)}
\PY{n}{numero1}\PY{o}{=}\PY{n+nb}{float}\PY{p}{(}\PY{n+nb}{input}\PY{p}{(}\PY{l+s+s2}{\PYZdq{}}\PY{l+s+s2}{Inserisci il primo numero: }\PY{l+s+s2}{\PYZdq{}}\PY{p}{)}\PY{p}{)}
\PY{n}{numero2}\PY{o}{=}\PY{n+nb}{float}\PY{p}{(}\PY{n+nb}{input}\PY{p}{(}\PY{l+s+s2}{\PYZdq{}}\PY{l+s+s2}{Inserisci il secondo numero: }\PY{l+s+s2}{\PYZdq{}}\PY{p}{)}\PY{p}{)}
\PY{n}{sottrazione}\PY{o}{=}\PY{n}{numero1}\PY{o}{\PYZhy{}}\PY{n}{numero2}
\PY{n+nb}{print}\PY{p}{(}\PY{l+s+s2}{\PYZdq{}}\PY{l+s+s2}{La sottrazione dei due numeri è pari a:}\PY{l+s+s2}{\PYZdq{}}\PY{p}{,}\PY{n+nb}{float}\PY{p}{(}\PY{n}{sottrazione}\PY{p}{)}\PY{p}{)}
\end{Verbatim}
\end{tcolorbox}

    \begin{Verbatim}[commandchars=\\\{\}]
Ciao, attraverso questa calcolatrice puoi eseguire calcoli con le sottrazioni!
Inserisci il primo numero: 3.4
Inserisci il secondo numero: 8.2
La sottrazione dei due numeri è pari a: -4.799999999999999
    \end{Verbatim}

    Il codice qui sotto svolge l'operazione di moltiplicazione tra due
numeri (sempre float) inseriti dall'utente. Funziona allo stesso modo ed
infatti ha una identica struttura tranne per il fatto che c'è il simbolo
* (asterisco) che in programmazione, non solo in Python ma anche nella
maggiorparte dei linguaggi, vuole significare il simbolo per (×) che fa
eseguire corretamente l'operazione in questione.

    \begin{tcolorbox}[breakable, size=fbox, boxrule=1pt, pad at break*=1mm,colback=cellbackground, colframe=cellborder]
\prompt{In}{incolor}{3}{\boxspacing}
\begin{Verbatim}[commandchars=\\\{\}]
\PY{c+c1}{\PYZsh{}Comprendere come eseguire le moltiplicazioni}
\PY{n+nb}{print}\PY{p}{(}\PY{l+s+s2}{\PYZdq{}}\PY{l+s+s2}{Ciao, attraverso questa calcolatrice puoi eseguire calcoli con le moltiplicazioni!}\PY{l+s+s2}{\PYZdq{}}\PY{p}{)}
\PY{n}{numero1}\PY{o}{=}\PY{n+nb}{float}\PY{p}{(}\PY{n+nb}{input}\PY{p}{(}\PY{l+s+s2}{\PYZdq{}}\PY{l+s+s2}{Inserisci il primo numero: }\PY{l+s+s2}{\PYZdq{}}\PY{p}{)}\PY{p}{)}
\PY{n}{numero2}\PY{o}{=}\PY{n+nb}{float}\PY{p}{(}\PY{n+nb}{input}\PY{p}{(}\PY{l+s+s2}{\PYZdq{}}\PY{l+s+s2}{Inserisci il secondo numero: }\PY{l+s+s2}{\PYZdq{}}\PY{p}{)}\PY{p}{)}
\PY{n}{moltiplicazione}\PY{o}{=}\PY{n}{numero1}\PY{o}{*}\PY{n}{numero2}\PY{c+c1}{\PYZsh{}il simbolo * vuol dire moltiplicazione}
\PY{n+nb}{print}\PY{p}{(}\PY{l+s+s2}{\PYZdq{}}\PY{l+s+s2}{La moltiplicazione dei due numeri è pari a:}\PY{l+s+s2}{\PYZdq{}}\PY{p}{,}\PY{n+nb}{float}\PY{p}{(}\PY{n}{moltiplicazione}\PY{p}{)}\PY{p}{)}
\end{Verbatim}
\end{tcolorbox}

    \begin{Verbatim}[commandchars=\\\{\}]
Ciao, attraverso questa calcolatrice puoi eseguire calcoli con le
moltiplicazioni!
Inserisci il primo numero: 3.9
Inserisci il secondo numero: 9.1
La moltiplicazione dei due numeri è pari a: 35.489999999999995
    \end{Verbatim}

    Il codice qui sotto svolge l'operazione di divisione tra due numeri
(sempre float) inseriti dall'utente. Funziona allo stesso modo ed
infatti ha una identica struttura tranne per il fatto che c'è il simbolo
/ (barra o slash in inglese) che in programmazione, non solo in Python
ma anche nella maggiorparte dei linguaggi, vuole significare il simbolo
della divisione (cioè il diviso) (: oppure il ÷) che fa eseguire
corretamente l'operazione in questione. Anche se si può dire che in
algebra o semplicemente in matematica il simbolo / viene usato per
scrivere le frazioni, che sono comunque delle divisoni a tutti gli
effetti.

    \begin{tcolorbox}[breakable, size=fbox, boxrule=1pt, pad at break*=1mm,colback=cellbackground, colframe=cellborder]
\prompt{In}{incolor}{1}{\boxspacing}
\begin{Verbatim}[commandchars=\\\{\}]
\PY{c+c1}{\PYZsh{}comprendere come eseguire le divisioni}
\PY{n+nb}{print}\PY{p}{(}\PY{l+s+s2}{\PYZdq{}}\PY{l+s+s2}{Ciao, attraverso questa calcolatrice puoi eseguire calcoli con le divisioni!}\PY{l+s+s2}{\PYZdq{}}\PY{p}{)}
\PY{n}{numero1}\PY{o}{=}\PY{n+nb}{float}\PY{p}{(}\PY{n+nb}{input}\PY{p}{(}\PY{l+s+s2}{\PYZdq{}}\PY{l+s+s2}{Inserisci il primo numero: }\PY{l+s+s2}{\PYZdq{}}\PY{p}{)}\PY{p}{)}
\PY{n}{numero2}\PY{o}{=}\PY{n+nb}{float}\PY{p}{(}\PY{n+nb}{input}\PY{p}{(}\PY{l+s+s2}{\PYZdq{}}\PY{l+s+s2}{Inserisci il secondo numero: }\PY{l+s+s2}{\PYZdq{}}\PY{p}{)}\PY{p}{)}
\PY{n}{divisione}\PY{o}{=}\PY{n}{numero1}\PY{o}{/}\PY{n}{numero2}\PY{c+c1}{\PYZsh{}il simbolo / vuol dire divisione}
\PY{n+nb}{print}\PY{p}{(}\PY{l+s+s2}{\PYZdq{}}\PY{l+s+s2}{La divisione dei due numeri è pari a:}\PY{l+s+s2}{\PYZdq{}}\PY{p}{,}\PY{n+nb}{float}\PY{p}{(}\PY{n}{divisione}\PY{p}{)}\PY{p}{)}
\end{Verbatim}
\end{tcolorbox}

    \begin{Verbatim}[commandchars=\\\{\}]
Ciao, attraverso questa calcolatrice puoi eseguire calcoli con le divisioni!
Inserisci il primo numero: 5.7
Inserisci il secondo numero: 2.6
La divisione dei due numeri è pari a: 2.1923076923076925
    \end{Verbatim}

    Il codice qui sotto svolge l'operazione di potenza tra due numeri (la
base come float e l'esponente come int) inseriti dall'utente. Funziona
allo stesso modo ed infatti ha una identica struttura tranne per il
fatto che c'è il simbolo ** (doppio asterisco, che letteralmente
vorrebbe dire doppio ×) che in programmazione, non solo in Python ma
anche nella maggiorparte dei linguaggi, vuole significare il simbolo
della potenza (x\^{}y, come il ** se fosse il \^{}) che fa eseguire
corretamente l'operazione in questione.

    \begin{tcolorbox}[breakable, size=fbox, boxrule=1pt, pad at break*=1mm,colback=cellbackground, colframe=cellborder]
\prompt{In}{incolor}{5}{\boxspacing}
\begin{Verbatim}[commandchars=\\\{\}]
\PY{c+c1}{\PYZsh{}Comprendere come eseguire le potenze}
\PY{n+nb}{print}\PY{p}{(}\PY{l+s+s2}{\PYZdq{}}\PY{l+s+s2}{Ciao, attraverso questa calcolatrice puoi calcolare il valore di una potenza!}\PY{l+s+s2}{\PYZdq{}}\PY{p}{)}
\PY{n}{base}\PY{o}{=}\PY{n+nb}{float}\PY{p}{(}\PY{n+nb}{input}\PY{p}{(}\PY{l+s+s2}{\PYZdq{}}\PY{l+s+s2}{Inserisci la base della potenza: }\PY{l+s+s2}{\PYZdq{}}\PY{p}{)}\PY{p}{)}\PY{c+c1}{\PYZsh{}la base è un float, questo perchè si può fare la potenza di un numero decimale}
\PY{n}{esponente}\PY{o}{=}\PY{n+nb}{int}\PY{p}{(}\PY{n+nb}{input}\PY{p}{(}\PY{l+s+s2}{\PYZdq{}}\PY{l+s+s2}{Inserisci l}\PY{l+s+s2}{\PYZsq{}}\PY{l+s+s2}{esponente della potenza: }\PY{l+s+s2}{\PYZdq{}}\PY{p}{)}\PY{p}{)}\PY{c+c1}{\PYZsh{}l\PYZsq{}esponente è un int invece, questo perchè per fare un esempio più facile è meglio usare l\PYZsq{}int in modo che il risultato sia più semplice}
\PY{n}{potenza}\PY{o}{=}\PY{n}{base}\PY{o}{*}\PY{o}{*}\PY{n}{esponente}\PY{c+c1}{\PYZsh{}il simbolo ** vuol dire potenza}
\PY{n+nb}{print}\PY{p}{(}\PY{l+s+s2}{\PYZdq{}}\PY{l+s+s2}{Il valore della potenza è pari a:}\PY{l+s+s2}{\PYZdq{}}\PY{p}{,}\PY{n+nb}{float}\PY{p}{(}\PY{n}{potenza}\PY{p}{)}\PY{p}{)}
\end{Verbatim}
\end{tcolorbox}

    \begin{Verbatim}[commandchars=\\\{\}]
Ciao, attraverso questa calcolatrice puoi calcolare il valore di una potenza!
Inserisci la base della potenza: 5.7
Inserisci l'esponente della potenza: 3
Il valore della potenza è pari a: 185.193
    \end{Verbatim}

    Questo codice riassume tutti i precedenti in un'unico codice. Il codice
infatti rappresenta una calcolatrice che consente all'utente di eseguire
operazioni di addizione, sottrazione, moltiplicazione, divisione e
calcolo delle potenze. Questa versatilità è ottenuta attraverso
l'utilizzo di condizioni (che in ordine gerarchico sono: if, elif ed
else) che dirigono il flusso del programma in base all'operazione scelta
dall'utente. Inizialmente, il programma richiede all'utente di inserire
il tipo di operazione che desidera effettuare sottoforma di segno,
nonché poi i due numeri coinvolti. L'input dell'utente viene
successivamente convertito in numeri in virgola decimale (i cosidetti
float) per consentire l'elaborazione di numeri decimali. La parte
principale del codice è pero costituita dalle condizioni if, elif ed
else (in scala gerarchica). Ciascuna condizione verifica se l'operatore
scritto nell'input corrisponde a uno specifico caso dichiarato in
precedenza (addizione, sottrazione, moltiplicazione, divisione, potenza)
e, in caso affermativo, esegue il blocco di codice associato a quella
condizione, cioè il blocco di codice scritto sotto alla condizione
finchè non incomincia una nuova condizione perchè poi a quel punto il
codice salta quel pezzo delle condizioni e va alla parte dopo. Se
l'operazione inserita non corrisponde a nessun caso noto, il blocco
nell'else fornisce un messaggio di errore. Inoltre l'else è sempre
l'ultima condizione mentre la prima è sempre l'if e quelle in mezzo sono
sempre l'elif. Questo approccio rende il codice estremamente flessibile,
consentendo allo sviluppatore di eseguire diverse operazioni senza dover
scrivere del codice separato per ciascuna di esse, utilizzando per
l'appunto le condizioni. Le condizioni giocano un ruolo fondamentale
nell'indirizzare il flusso del programma in modo dinamico, in base alle
scelte dell'utente, offrendo così in questo caso una calcolatrice
completa e versatile.

    \begin{tcolorbox}[breakable, size=fbox, boxrule=1pt, pad at break*=1mm,colback=cellbackground, colframe=cellborder]
\prompt{In}{incolor}{1}{\boxspacing}
\begin{Verbatim}[commandchars=\\\{\}]
\PY{c+c1}{\PYZsh{}Esempio di calcolatrice finale con tutte cinque le operazioni}
\PY{n+nb}{print}\PY{p}{(}\PY{l+s+s2}{\PYZdq{}}\PY{l+s+s2}{Ciao, attraverso questa calcolatrice puoi eseguire calcoli con le addizioni, sottrazioni, moltiplicazioni e divisioni, inoltre puoi anche calcolare il valore di una potenza!}\PY{l+s+s2}{\PYZdq{}}\PY{p}{)}
\PY{n}{operazione}\PY{o}{=}\PY{n+nb}{input}\PY{p}{(}\PY{l+s+s2}{\PYZdq{}}\PY{l+s+s2}{Adesso scegli l}\PY{l+s+s2}{\PYZsq{}}\PY{l+s+s2}{operazione che fa al caso tuo attraverso il correlato simbolo (+, \PYZhy{}, *, /, **): }\PY{l+s+s2}{\PYZdq{}}\PY{p}{)}
\PY{n}{numero1}\PY{o}{=}\PY{n+nb}{float}\PY{p}{(}\PY{n+nb}{input}\PY{p}{(}\PY{l+s+s2}{\PYZdq{}}\PY{l+s+s2}{Inserisci il primo numero o la base della potenza: }\PY{l+s+s2}{\PYZdq{}}\PY{p}{)}\PY{p}{)}
\PY{n}{numero2}\PY{o}{=}\PY{n+nb}{float}\PY{p}{(}\PY{n+nb}{input}\PY{p}{(}\PY{l+s+s2}{\PYZdq{}}\PY{l+s+s2}{Inserisci il secondo numero o l}\PY{l+s+s2}{\PYZsq{}}\PY{l+s+s2}{esponente della potenza: }\PY{l+s+s2}{\PYZdq{}}\PY{p}{)}\PY{p}{)}
\PY{k}{if} \PY{n}{operazione}\PY{o}{==}\PY{l+s+s2}{\PYZdq{}}\PY{l+s+s2}{+}\PY{l+s+s2}{\PYZdq{}}\PY{p}{:}\PY{c+c1}{\PYZsh{}l\PYZsq{}if indica il primo caso possibile}
    \PY{n}{risultato}\PY{o}{=}\PY{n}{numero1}\PY{o}{+}\PY{n}{numero2}
\PY{k}{elif} \PY{n}{operazione}\PY{o}{==}\PY{l+s+s2}{\PYZdq{}}\PY{l+s+s2}{\PYZhy{}}\PY{l+s+s2}{\PYZdq{}}\PY{p}{:}\PY{c+c1}{\PYZsh{}\PYZsh{}l\PYZsq{}elif è l\PYZsq{}ibrido tra if \PYZam{} else e indica i casi di mezzo}
    \PY{n}{risultato}\PY{o}{=}\PY{n}{numero1}\PY{o}{\PYZhy{}}\PY{n}{numero2}
\PY{k}{elif} \PY{n}{operazione}\PY{o}{==}\PY{l+s+s2}{\PYZdq{}}\PY{l+s+s2}{*}\PY{l+s+s2}{\PYZdq{}}\PY{p}{:}
    \PY{n}{risultato}\PY{o}{=}\PY{n}{numero1}\PY{o}{*}\PY{n}{numero2}
\PY{k}{elif} \PY{n}{operazione}\PY{o}{==}\PY{l+s+s2}{\PYZdq{}}\PY{l+s+s2}{/}\PY{l+s+s2}{\PYZdq{}}\PY{p}{:}
    \PY{n}{risultato}\PY{o}{=}\PY{n}{numero1}\PY{o}{/}\PY{n}{numero2}
\PY{k}{elif} \PY{n}{operazione}\PY{o}{==}\PY{l+s+s2}{\PYZdq{}}\PY{l+s+s2}{**}\PY{l+s+s2}{\PYZdq{}}\PY{p}{:}
    \PY{n}{risultato}\PY{o}{=}\PY{n}{numero1}\PY{o}{*}\PY{o}{*}\PY{n}{numero2}
\PY{k}{else}\PY{p}{:}\PY{c+c1}{\PYZsh{}l\PYZsq{}else indica l\PYZsq{}ultimo caso possibile}
    \PY{n}{risultato}\PY{o}{=}\PY{l+s+s2}{\PYZdq{}}\PY{l+s+s2}{operatore non riconosciuto. Scegli uno dei seguenti operatori riavviando prima il programma: +, \PYZhy{}, *, /, **}\PY{l+s+s2}{\PYZdq{}}
\PY{n+nb}{print}\PY{p}{(}\PY{l+s+s2}{\PYZdq{}}\PY{l+s+s2}{Il risultato finale del calcolo è:}\PY{l+s+s2}{\PYZdq{}}\PY{p}{,}\PY{n}{risultato}\PY{p}{)}
\end{Verbatim}
\end{tcolorbox}

    \begin{Verbatim}[commandchars=\\\{\}]
Ciao, attraverso questa calcolatrice puoi eseguire calcoli con le addizioni,
sottrazioni, moltiplicazioni e divisioni, inoltre puoi anche calcolare il valore
di una potenza!
Adesso scegli l'operazione che fa al caso tuo attraverso il correlato simbolo
(+, -, *, /, **): non lo so
Inserisci il primo numero o la base della potenza: 5
Inserisci il secondo numero o l'esponente della potenza: 5
Il risultato finale del calcolo è: operatore non riconosciuto. Scegli uno dei
seguenti operatori riavviando prima il programma: +, -, *, /, **
    \end{Verbatim}

    \section{IL CICLO FOR CON N}\label{il-ciclo-for-con-n}

    Il programma in questione è un semplice ``generatore'' di numeri che
parte e arriva fino a un numero desiderato dall'utente. Inizia chiedendo
all'utente due numeri positivi: il primo (numerodaverificare) indica
fino a quale numero si vuole generare la lista della sequenza di numeri,
mentre il secondo (numeroiniziale) rappresenta da quale numero positivo
si desidera iniziare la numerazione. Successivamente, attraverso un
ciclo for, il programma itera attraverso una sequenza di numeri a
partire da ``numeroiniziale'' fino a ``numerodaverificare'' incluso.
Durante ogni iterazione del ciclo, il numero corrente viene stampato a
schermo dopo una stringa esplicativa ed è così per ogni volta che si
ripete questo ciclo. Questo processo continua fino a quando il ciclo
raggiunge il valore scelto dall'utente della variabile
``numerodaverificare''. Quindi alla fine si può dire che l'uso di
range(numeroiniziale, numerodaverificare + 1) assicura che il valore
finale ``numeroiniziale'' sia incluso nella sequenza, poiché senza il
+1, la sequenza generata terminerebbe a ``numeroiniziale'' - 1, cioè al
numero precedente da quello scelto dall'utente, quindi escludendo il
valore desiderato ``numerodaverificare'' dalla lista finale che viene
stampata. In questo modo, il +1 garantisce che la numerazione vada da
``numeroiniziale'' fino a ``numerodaverificare'' scrivendo tutti i
numeri della lista, inclusi entrambi i ``poli'' decisi come input
dall'utente. Questo tipo di programma è utile quando si desidera
generare e visualizzare una sequenza ordinata di numeri in base, però,
alle preferenze dell'utente.

    \begin{tcolorbox}[breakable, size=fbox, boxrule=1pt, pad at break*=1mm,colback=cellbackground, colframe=cellborder]
\prompt{In}{incolor}{5}{\boxspacing}
\begin{Verbatim}[commandchars=\\\{\}]
\PY{c+c1}{\PYZsh{}Comprendere il ciclo for con n}
\PY{n+nb}{print}\PY{p}{(}\PY{l+s+s2}{\PYZdq{}}\PY{l+s+s2}{Ciao, attraverso questo programma puoi verificare la numerazione ordinaria dei numeri dal e fino al punto che desideri}\PY{l+s+s2}{\PYZdq{}}\PY{p}{)}
\PY{n}{numerodaverificare}\PY{o}{=}\PY{n+nb}{int}\PY{p}{(}\PY{n+nb}{input}\PY{p}{(}\PY{l+s+s2}{\PYZdq{}}\PY{l+s+s2}{Quindi, inserisci il numero positivo che desideri verificare: }\PY{l+s+s2}{\PYZdq{}}\PY{p}{)}\PY{p}{)}
\PY{n}{numeroiniziale}\PY{o}{=}\PY{n+nb}{int}\PY{p}{(}\PY{n+nb}{input}\PY{p}{(}\PY{l+s+s2}{\PYZdq{}}\PY{l+s+s2}{Poi, inserisci da quale numero positivo desideri che la numerazione inizii: }\PY{l+s+s2}{\PYZdq{}}\PY{p}{)}\PY{p}{)}
\PY{n+nb}{print}\PY{p}{(}\PY{l+s+s2}{\PYZdq{}}\PY{l+s+s2}{La numerazione ordinaria dei numeri dal e fino al punto che desideri è:}\PY{l+s+s2}{\PYZdq{}}\PY{p}{)}
\PY{k}{for} \PY{n}{numero} \PY{o+ow}{in} \PY{n+nb}{range}\PY{p}{(}\PY{n}{numeroiniziale}\PY{p}{,}\PY{n}{numerodaverificare}\PY{o}{+}\PY{l+m+mi}{1}\PY{p}{)}\PY{p}{:}\PY{c+c1}{\PYZsh{}p è un valore fisso determinato dall\PYZsq{}utente mentre +1 indica che a n verrà aggiunto il valore di 1. N.B.: la variabile numero è diversa dalle altre due infatti rappresenta ciascun numero nella numerazione ordinaria tra numeroiniziale e numerodaverificare, cioè assume successivamente ogni valore compreso tra numeroiniziale e numerodaverificare e non ce mai bisogno di inizzalizzarla in Python}
    \PY{n+nb}{print}\PY{p}{(}\PY{n}{numero}\PY{p}{)}
\end{Verbatim}
\end{tcolorbox}

    \begin{Verbatim}[commandchars=\\\{\}]
Ciao, attraverso questo programma puoi verificare la numerazione ordinaria dei
numeri dal e fino al punto che desideri
Quindi, inserisci il numero positivo che desideri verificare: 33
Poi, inserisci da quale numero positivo desideri che la numerazione inizii: 12
La numerazione ordinaria dei numeri dal e fino al punto che desideri è:
12
13
14
15
16
17
18
19
20
21
22
23
24
25
26
27
28
29
30
31
32
33
    \end{Verbatim}

    Il programma sottostante calcola la somma cumulativa dei primi n numeri
interi positivi chiedendo il numero desiderato all'utente come input. La
somma cumulativa è un concetto matematico che rappresenta la somma
progressiva di una sequenza di numeri, aggiungendo ad ogni passo il
numero corrente all'accumulatore. In questo contesto, il programma
calcola la somma cumulativa dei primi n numeri interi positivi, fornendo
un risultato che riflette l'incremento graduale della somma durante
l'iterazione. Ad esempio, se l'utente inserisce n=5, la somma cumulativa
sarà calcolata come 1+2+3+4+5, che darà come risultato 15. Il programma
inizia chiedendo all'utente di inserire come input un numero intero
positivo (variabile n) di cui desidera calcolare la somma cumulativa.
Successivamente, il programma utilizza un ciclo for per iterare
attraverso i numeri da 1 a n inclusi e quindi dal numero 1 fino al
numero scelto dall'utente. Durante ogni iterazione, il programma
aggiunge il numero corrente alla variabile addizione, che funge da
accumulatore per la somma cumulativa, finchè il processo di iterazione
non sarà arrivato al numero scelto dall'utente. L'operatore += usato in
questo codice ad ogni ciclo di iterazione esegue contemporaneamente
l'addizione e la riassegnazione del risultato alla variabile addizione,
cioè per prima cosa fa il calcolo dell'addizione che deve svolgere e poi
riassegna il valore del risultato di quel calcolo alla variabile
addizione e questo permette di far due azione che dovrebbero essere
assegnate con due comandi diversi in uno singolo in modo da risparmiare
memoria. Infine, il programma stampa il risultato, indicando la somma
dei primi n numeri interi e sottolineando che questo valore rappresenta
la somma cumulativa. Questo codice fornisce un'implementazione semplice
della somma cumulativa.

    \begin{tcolorbox}[breakable, size=fbox, boxrule=1pt, pad at break*=1mm,colback=cellbackground, colframe=cellborder]
\prompt{In}{incolor}{3}{\boxspacing}
\begin{Verbatim}[commandchars=\\\{\}]
\PY{c+c1}{\PYZsh{}Comprendere il ciclo for con n facendo la somma cumulativa}
\PY{n+nb}{print}\PY{p}{(}\PY{l+s+s2}{\PYZdq{}}\PY{l+s+s2}{Ciao, attraverso questo programma puoi calcolare la somma cumulativa di qualsiasi numero che desideri sapere}\PY{l+s+s2}{\PYZdq{}}\PY{p}{)}
\PY{n}{n}\PY{o}{=}\PY{n+nb}{int}\PY{p}{(}\PY{n+nb}{input}\PY{p}{(}\PY{l+s+s2}{\PYZdq{}}\PY{l+s+s2}{Quindi, inserisci il numero positivo che desideri calcolare: }\PY{l+s+s2}{\PYZdq{}}\PY{p}{)}\PY{p}{)}
\PY{n}{addizione}\PY{o}{=}\PY{l+m+mi}{0}
\PY{k}{for} \PY{n}{numero} \PY{o+ow}{in} \PY{n+nb}{range}\PY{p}{(}\PY{l+m+mi}{1}\PY{p}{,}\PY{n}{n}\PY{o}{+}\PY{l+m+mi}{1}\PY{p}{)}\PY{p}{:}\PY{c+c1}{\PYZsh{}1 è un valore fisso mentre +1 indica che a n verrà aggiunto il valore di 1}
    \PY{n}{addizione}\PY{o}{+}\PY{o}{=}\PY{n}{numero}\PY{c+c1}{\PYZsh{}il simbolo += svolge due compiti in uno: fa la somma e riassegna il valore della variabile}
\PY{n+nb}{print}\PY{p}{(}\PY{l+s+s2}{\PYZdq{}}\PY{l+s+s2}{La somma dei primi}\PY{l+s+s2}{\PYZdq{}}\PY{p}{,}\PY{n}{n}\PY{p}{,}\PY{l+s+s2}{\PYZdq{}}\PY{l+s+s2}{numeri interi è}\PY{l+s+s2}{\PYZdq{}}\PY{p}{,}\PY{n}{addizione}\PY{p}{)}
\PY{n+nb}{print}\PY{p}{(}\PY{l+s+s2}{\PYZdq{}}\PY{l+s+s2}{Perciò la somma cumulativa di}\PY{l+s+s2}{\PYZdq{}}\PY{p}{,}\PY{n}{n}\PY{p}{,}\PY{l+s+s2}{\PYZdq{}}\PY{l+s+s2}{è}\PY{l+s+s2}{\PYZdq{}}\PY{p}{,}\PY{n}{addizione}\PY{p}{)}
\end{Verbatim}
\end{tcolorbox}

    \begin{Verbatim}[commandchars=\\\{\}]
Ciao, attraverso questo programma puoi calcolare la somma cumulativa di
qualsiasi numero che desideri sapere
Quindi, inserisci il numero positivo che desideri calcolare: 12
La somma dei primi 12 numeri interi è 78
Perciò la somma cumulativa di 12 è 78
    \end{Verbatim}

    \section{CALCOLARE LA MEDIA DI UNA LISTA DI
NUMERI}\label{calcolare-la-media-di-una-lista-di-numeri}

    Questo programma consente all'utente di calcolare la media di una lista
di numeri scelta dall'utente stesso. La media dei numeri è ottenuta
sommando tutti i numeri insieme e dividendo il risultato per il numero
totale di numeri. Inizialmente, il programma chiede all'utente quanti
numeri desidera inserire nella lista. Utilizzando poi un ciclo for, il
programma acquisisce ogni numero specifico della lista inserito
dall'utente e li aggiunge alla lista numeri utilizzando il metodo
append, che per l'appunto ``appende'' ognuno dei numeri scritti
all'utente e salvati nella variabile ``numero'' alla lista vera e
propria del programma. Successivamente, viene calcolata la media dei
numeri presenti nella lista utilizzando i comandi sum e len. Il comando
sum(numeri) restituisce la somma di tutti i numeri nella lista, mentre
len(numeri) restituisce la lunghezza della lista, cioè il numero totale
di numeri inseriti. Dividendo la somma per la lunghezza, si ottiene così
la media dei numeri. Infine il programma stampa la lista finale,
fornendo così all'utente la media di tutti i numeri inseriti da lui.
Questo codice fornisce un modo efficiente per calcolare la media di una
lista di numeri inseriti dall'utente.

    \begin{tcolorbox}[breakable, size=fbox, boxrule=1pt, pad at break*=1mm,colback=cellbackground, colframe=cellborder]
\prompt{In}{incolor}{1}{\boxspacing}
\begin{Verbatim}[commandchars=\\\{\}]
\PY{c+c1}{\PYZsh{}Comprendere come usare i comandi sum e len facendo la media di una lista di numeri}
\PY{n}{numeri}\PY{o}{=}\PY{p}{[}\PY{p}{]}\PY{c+c1}{\PYZsh{}l\PYZsq{}append serve a fare una lista da poter richiamare}
\PY{n+nb}{print}\PY{p}{(}\PY{l+s+s2}{\PYZdq{}}\PY{l+s+s2}{Ciao, attraverso questo programma puoi calcolare la media di una lista di numeri che desideri sapere}\PY{l+s+s2}{\PYZdq{}}\PY{p}{)}
\PY{n}{n}\PY{o}{=}\PY{n+nb}{int}\PY{p}{(}\PY{n+nb}{input}\PY{p}{(}\PY{l+s+s2}{\PYZdq{}}\PY{l+s+s2}{Quanti numeri desideri inserire in totale? }\PY{l+s+s2}{\PYZdq{}}\PY{p}{)}\PY{p}{)}
\PY{k}{for} \PY{n}{i} \PY{o+ow}{in} \PY{n+nb}{range}\PY{p}{(}\PY{n}{n}\PY{p}{)}\PY{p}{:}
    \PY{n}{numero}\PY{o}{=}\PY{n+nb}{float}\PY{p}{(}\PY{n+nb}{input}\PY{p}{(}\PY{l+s+s2}{\PYZdq{}}\PY{l+s+s2}{Inserisci il numero di cui desideri fare la media: }\PY{l+s+s2}{\PYZdq{}}\PY{p}{)}\PY{p}{)}
    \PY{n}{numeri}\PY{o}{.}\PY{n}{append}\PY{p}{(}\PY{n}{numero}\PY{p}{)}\PY{c+c1}{\PYZsh{}così si crea la lista che varia dalla variabile numero, \PYZdq{}appendendo\PYZdq{} il numero scritto dall\PYZsq{}utente prima alla lista vera e propria}
\PY{n}{mediadeinumeri}\PY{o}{=}\PY{n+nb}{sum}\PY{p}{(}\PY{n}{numeri}\PY{p}{)}\PY{o}{/}\PY{n+nb}{len}\PY{p}{(}\PY{n}{numeri}\PY{p}{)}\PY{c+c1}{\PYZsh{}sum=somma mentre len=lunghezza}
\PY{n+nb}{print}\PY{p}{(}\PY{l+s+s2}{\PYZdq{}}\PY{l+s+s2}{La media di tutti i numeri inseriti è:}\PY{l+s+s2}{\PYZdq{}}\PY{p}{,}\PY{n}{mediadeinumeri}\PY{p}{)}
\end{Verbatim}
\end{tcolorbox}

    \begin{Verbatim}[commandchars=\\\{\}]
Ciao, attraverso questo programma puoi calcolare la media di una lista di numeri
che desideri sapere
Quanti numeri desideri inserire in totale? 5
Inserisci il numero di cui desideri fare la media: 12
Inserisci il numero di cui desideri fare la media: 33
Inserisci il numero di cui desideri fare la media: 57
Inserisci il numero di cui desideri fare la media: 127
Inserisci il numero di cui desideri fare la media: 73
La media di tutti i numeri inseriti è: 60.4
    \end{Verbatim}

    \section{CALCOLARE IL QUADRATO DEI
NUMERI}\label{calcolare-il-quadrato-dei-numeri}

    Il programma consente all'utente di calcolare il quadrato di un numero
specificato sempre dall'utente. Il quadrato di un numero è ottenuto
moltiplicando il numero per se stesso, per l'appunto facendo la potenza
alla seconda del numero in questione. Inizialmente, il programma
richiede all'utente di inserire come input un numero intero (variabile
n) di cui desidera calcolare il suo quadrato. Successivamente,
attraverso un ciclo for, il programma itera attraverso i numeri da 1 a n
inclusi. Per ogni numero da 1 a n, il ciclo for calcola il quadrato
utilizzando l'operatore di potenza ** e poi stampa il risultato dopo una
stringa di testo esplicativa. Il risultato finale di output è una lista
di quadrati dei primi n numeri, con ogni riga che mostra all'utente il
numero originale e il suo quadrato corrispondente. Questo codice
fornisce un modo semplice per calcolare il quadrato di un numero.

    \begin{tcolorbox}[breakable, size=fbox, boxrule=1pt, pad at break*=1mm,colback=cellbackground, colframe=cellborder]
\prompt{In}{incolor}{1}{\boxspacing}
\begin{Verbatim}[commandchars=\\\{\}]
\PY{c+c1}{\PYZsh{}Comprendere come eseguire le potenze facendo il quadrato dei numeri}
\PY{n+nb}{print}\PY{p}{(}\PY{l+s+s2}{\PYZdq{}}\PY{l+s+s2}{Ciao, attraverso questo programma puoi calcolare il quadrato di un numero che desideri sapere}\PY{l+s+s2}{\PYZdq{}}\PY{p}{)}
\PY{n}{n}\PY{o}{=}\PY{n+nb}{int}\PY{p}{(}\PY{n+nb}{input}\PY{p}{(}\PY{l+s+s2}{\PYZdq{}}\PY{l+s+s2}{Quindi, inserisci il numero di cui desideri sapere il quadrato: }\PY{l+s+s2}{\PYZdq{}}\PY{p}{)}\PY{p}{)}
\PY{n+nb}{print}\PY{p}{(}\PY{l+s+s2}{\PYZdq{}}\PY{l+s+s2}{I quadrati dei primi}\PY{l+s+s2}{\PYZdq{}}\PY{p}{,}\PY{n}{n}\PY{p}{,}\PY{l+s+s2}{\PYZdq{}}\PY{l+s+s2}{numeri sono: }\PY{l+s+s2}{\PYZdq{}}\PY{p}{)}
\PY{k}{for} \PY{n}{numero} \PY{o+ow}{in} \PY{n+nb}{range}\PY{p}{(}\PY{l+m+mi}{1}\PY{p}{,}\PY{n}{n}\PY{o}{+}\PY{l+m+mi}{1}\PY{p}{)}\PY{p}{:}
    \PY{n}{quadratodelnumero}\PY{o}{=}\PY{n}{numero}\PY{o}{*}\PY{o}{*}\PY{l+m+mi}{2}\PY{c+c1}{\PYZsh{}il simbolo di potenza si fa con il doppio **}
    \PY{n+nb}{print}\PY{p}{(}\PY{l+s+s2}{\PYZdq{}}\PY{l+s+s2}{Il quadrato di}\PY{l+s+s2}{\PYZdq{}}\PY{p}{,}\PY{n}{numero}\PY{p}{,}\PY{l+s+s2}{\PYZdq{}}\PY{l+s+s2}{è}\PY{l+s+s2}{\PYZdq{}}\PY{p}{,}\PY{n}{quadratodelnumero}\PY{p}{)}
\end{Verbatim}
\end{tcolorbox}

    \begin{Verbatim}[commandchars=\\\{\}]
Ciao, attraverso questo programma puoi calcolare il quadrato di un numero che
desideri sapere
Quindi, inserisci il numero di cui desideri sapere il quadrato: 12
I quadrati dei primi 12 numeri sono:
Il quadrato di 1 è 1
Il quadrato di 2 è 4
Il quadrato di 3 è 9
Il quadrato di 4 è 16
Il quadrato di 5 è 25
Il quadrato di 6 è 36
Il quadrato di 7 è 49
Il quadrato di 8 è 64
Il quadrato di 9 è 81
Il quadrato di 10 è 100
Il quadrato di 11 è 121
Il quadrato di 12 è 144
    \end{Verbatim}

    \section{LA VERIFICA DI PARITÀ O DISPARITÀ DI UN
NUMERO}\label{la-verifica-di-parituxe0-o-disparituxe0-di-un-numero}

    Il programma consente all'utente di verificare se un numero inserito
come input è pari o dispari. Inizialmente, chiede all'utente di inserire
un numero intero come input della variabile numero. Successivamente,
utilizza l'operatore modulo ``\%'' (simbolo di percentuale) per
verificare se il numero è divisibile per 2 o meno. Se il resto della
divisione di numero per 2 è uguale a 0, il programma stampa che il
numero è pari. In caso contrario, se il resto è diverso da 0, il
programma stampa che il numero è dispari. L'operatore modulo restituisce
il resto della divisione tra i due operandi, cioè in questo caso il
numero definito dall'utente e il 2. Nello specifico nella condizione
``if numero\%2==0:'', il programma verifica se il numero è divisibile
per 2 (cioè se il resto è 0), il che indica che il numero è pari. L'else
ovviamente, se il programma si trova nella altra condizione, stampa che
è dispari. Questo codice fornisce una semplice verifica della parità o
disparità di un numero.

    \begin{tcolorbox}[breakable, size=fbox, boxrule=1pt, pad at break*=1mm,colback=cellbackground, colframe=cellborder]
\prompt{In}{incolor}{2}{\boxspacing}
\begin{Verbatim}[commandchars=\\\{\}]
\PY{c+c1}{\PYZsh{}Comprendere l\PYZsq{}utillizo del simbolo di \PYZpc{}}
\PY{n+nb}{print}\PY{p}{(}\PY{l+s+s2}{\PYZdq{}}\PY{l+s+s2}{Ciao, attraverso questo programma puoi verificare se un numero è pari o dispari}\PY{l+s+s2}{\PYZdq{}}\PY{p}{)}
\PY{n}{numero}\PY{o}{=}\PY{n+nb}{int}\PY{p}{(}\PY{n+nb}{input}\PY{p}{(}\PY{l+s+s2}{\PYZdq{}}\PY{l+s+s2}{Quindi, inserisci il numero che desideri verificare: }\PY{l+s+s2}{\PYZdq{}}\PY{p}{)}\PY{p}{)}
\PY{k}{if} \PY{n}{numero}\PY{o}{\PYZpc{}}\PY{k}{2}==0:\PYZsh{}il simbolo di \PYZpc{} vuol dire se è divisibile o meno un numero, ==0 vuol dire se in quel caso è con il resto di 0
    \PY{n+nb}{print}\PY{p}{(}\PY{l+s+s2}{\PYZdq{}}\PY{l+s+s2}{Il numero}\PY{l+s+s2}{\PYZdq{}}\PY{p}{,}\PY{n}{numero}\PY{p}{,}\PY{l+s+s2}{\PYZdq{}}\PY{l+s+s2}{è un numero pari}\PY{l+s+s2}{\PYZdq{}}\PY{p}{)}
\PY{k}{else}\PY{p}{:}
    \PY{n+nb}{print}\PY{p}{(}\PY{l+s+s2}{\PYZdq{}}\PY{l+s+s2}{Il numero}\PY{l+s+s2}{\PYZdq{}}\PY{p}{,}\PY{n}{numero}\PY{p}{,}\PY{l+s+s2}{\PYZdq{}}\PY{l+s+s2}{è un numero dispari}\PY{l+s+s2}{\PYZdq{}}\PY{p}{)}
\end{Verbatim}
\end{tcolorbox}

    \begin{Verbatim}[commandchars=\\\{\}]
Ciao, attraverso questo programma puoi verificare se un numero è pari o dispari
Quindi, inserisci il numero che desideri verificare: 12
Il numero 12 è un numero pari
    \end{Verbatim}

    \section{CALCOLARE IL FATTORIALE}\label{calcolare-il-fattoriale}

    Il programma qui sotto consente all'utente di calcolare il fattoriale di
un numero specificato dall'utente stesso. Il fattoriale di un numero è
ottenuto moltiplicando tutti i numeri interi positivi fino a quel
numero, il fattoriale di 5 ad esempio è calcolato moltiplicando tutti i
numeri interi positivi da 1 a 5, cioè: 1 * 2 * 3 * 4 * 5, che è uguale a
120. Inizialmente, il programma richiede all'utente come input di
inserire un numero positivo (variabile n) di cui desidera calcolare il
fattoriale. Successivamente, attraverso un ciclo for, il programma itera
attraverso i numeri da 1 a n inclusi. Per ogni numero, il ciclo for
moltiplica il valore corrente della variabile fattoriale per il numero
stesso e riassegna il risultato alla variabile fattoriale. L'operatore
\emph{= è un modo conciso e semplice per eseguire moltiplicazioni e
riassegnare il risultato alla stessa variabile, praticamente fa due cose
in uno: svolge la moltiplicazione e salva il risultato nella stessa
variabile senza l'uso di due comandi separati. Ad esempio,
fattoriale}=numero è equivalente a fattoriale=fattoriale*numero solo che
per quest'ultimo il programma userà più memoria. Alla fine del ciclo,
dopo una stringa esplicativa, il programma stampa il risultato che
rappresenta il fattoriale del numero inserito all'inizio dall'utente.

    \begin{tcolorbox}[breakable, size=fbox, boxrule=1pt, pad at break*=1mm,colback=cellbackground, colframe=cellborder]
\prompt{In}{incolor}{1}{\boxspacing}
\begin{Verbatim}[commandchars=\\\{\}]
\PY{c+c1}{\PYZsh{}Comprendere l\PYZsq{}utillizo del simbolo di *=}
\PY{n+nb}{print}\PY{p}{(}\PY{l+s+s2}{\PYZdq{}}\PY{l+s+s2}{Ciao, attraverso questo programma puoi calcolare il fattoriale di qualsiasi numero che desideri sapere}\PY{l+s+s2}{\PYZdq{}}\PY{p}{)}
\PY{n}{n}\PY{o}{=}\PY{n+nb}{int}\PY{p}{(}\PY{n+nb}{input}\PY{p}{(}\PY{l+s+s2}{\PYZdq{}}\PY{l+s+s2}{Quindi, inserisci il numero positivo che desideri calcolare: }\PY{l+s+s2}{\PYZdq{}}\PY{p}{)}\PY{p}{)}
\PY{n}{fattoriale}\PY{o}{=}\PY{l+m+mi}{1}
\PY{k}{for} \PY{n}{numero} \PY{o+ow}{in} \PY{n+nb}{range}\PY{p}{(}\PY{l+m+mi}{1}\PY{p}{,}\PY{n}{n}\PY{o}{+}\PY{l+m+mi}{1}\PY{p}{)}\PY{p}{:}
    \PY{n}{fattoriale}\PY{o}{*}\PY{o}{=}\PY{n}{numero}\PY{c+c1}{\PYZsh{}il simbolo *= svolge due compiti in uno: fa la moltiplicazione e riassegna il valore della variabile}
\PY{n+nb}{print}\PY{p}{(}\PY{l+s+s2}{\PYZdq{}}\PY{l+s+s2}{Il fattoriale del numero}\PY{l+s+s2}{\PYZdq{}}\PY{p}{,}\PY{n}{n}\PY{p}{,}\PY{l+s+s2}{\PYZdq{}}\PY{l+s+s2}{è}\PY{l+s+s2}{\PYZdq{}}\PY{p}{,}\PY{n}{fattoriale}\PY{p}{)}
\end{Verbatim}
\end{tcolorbox}

    \begin{Verbatim}[commandchars=\\\{\}]
Ciao, attraverso questo programma puoi calcolare il fattoriale di qualsiasi
numero che desideri sapere
Quindi, inserisci il numero positivo che desideri calcolare: 12
Il fattoriale del numero 12 è 479001600
    \end{Verbatim}

    \section{IL GIOCO DELL'INDOVINELLO}\label{il-gioco-dellindovinello}

    Il programma in questione è un semplice gioco in cui l'utente, o in
questo caso il giocatore, deve indovinare un numero generato casualmente
compreso tra 1 e 100. Inizialmente, il programma utilizza la libreria
random per generare un numero segreto (salvato nella variabile
numerodaindovinare) mediante la funzione randint(1, 100), che
restituisce un numero intero casuale compreso tra 1 e 100 (inclusi).
Successivamente, il programma inizia un ciclo ``while True'' che si
ripeterà finché l'utente non indovina il numero segreto. Il ciclo
``while True'' non è altro che un ciclo che dura fino all'``infinito'',
cioè fino a quando l'utente non si sarà imbattuto nell'unica condizione
del programma che attraverso il comando ``break'' riuscirà ad
interrompere questo ciclo continuo e far così terminare il programma; se
durante lo sviluppo del programma ci si dimentica di inserire questo
comando il programma potrebbe realmente continuare ``per sempre'', anche
se questo sarebbe scientificamente impossibile. Durante ogni iterazione
del ciclo, il programma chiede all'utente di inserire un tentativo
numerico (salvato nella variabile tentativo). Il programma tiene traccia
del numero di tentativi effetuati con la variabile ``tentativi''. Se
l'utente riesce ad indovinare il numero segreto, il programma stampa un
messaggio di congratulazioni insieme al numero segreto stampato a
schermo e al numero di tentativi effetuati, quindi a quel punto
quest'ultimo esce dal ciclo con l'istruzione ``break''. Se il tentativo
dell'utente è sbagliato, il programma fornisce un suggerimento sulla
direzione corretta per aiutare il giocatore (in particolare il programma
specifica se il numero è più grande o più piccolo, questo grazie a delle
condizioni specifiche inserite nel programma).

    \begin{tcolorbox}[breakable, size=fbox, boxrule=1pt, pad at break*=1mm,colback=cellbackground, colframe=cellborder]
\prompt{In}{incolor}{3}{\boxspacing}
\begin{Verbatim}[commandchars=\\\{\}]
\PY{c+c1}{\PYZsh{}Comprendere l\PYZsq{}utillizzo della libreria random}
\PY{k+kn}{import} \PY{n+nn}{random}
\PY{n}{numerodaindovinare}\PY{o}{=}\PY{n}{random}\PY{o}{.}\PY{n}{randint}\PY{p}{(}\PY{l+m+mi}{1}\PY{p}{,}\PY{l+m+mi}{100}\PY{p}{)}\PY{c+c1}{\PYZsh{}random.randint vuol dire che il numero da generare è compreso da 1 a 100 ed è intero perchè è int}
\PY{n}{tentativi}\PY{o}{=}\PY{l+m+mi}{0}
\PY{n+nb}{print}\PY{p}{(}\PY{l+s+s2}{\PYZdq{}}\PY{l+s+s2}{Ciao, questo gioco prevede che tu indovini un numero compreso da 1 a 100}\PY{l+s+s2}{\PYZdq{}}\PY{p}{)}
\PY{k}{while} \PY{k+kc}{True}\PY{p}{:}
    \PY{n}{tentativo}\PY{o}{=}\PY{n+nb}{int}\PY{p}{(}\PY{n+nb}{input}\PY{p}{(}\PY{l+s+s2}{\PYZdq{}}\PY{l+s+s2}{Adesso inserisci il numero che ritieni sia quello giusto: }\PY{l+s+s2}{\PYZdq{}}\PY{p}{)}\PY{p}{)}
    \PY{n}{tentativi} \PY{o}{+}\PY{o}{=} \PY{l+m+mi}{1}
    \PY{k}{if} \PY{n}{tentativo}\PY{o}{==}\PY{n}{numerodaindovinare}\PY{p}{:}
        \PY{n+nb}{print}\PY{p}{(}\PY{l+s+s2}{\PYZdq{}}\PY{l+s+s2}{Complimenti! Sei riuscito a indovinare il numero incognito che è}\PY{l+s+s2}{\PYZdq{}}\PY{p}{,}\PY{n}{numerodaindovinare}\PY{p}{,}\PY{l+s+s2}{\PYZdq{}}\PY{l+s+s2}{in}\PY{l+s+s2}{\PYZdq{}}\PY{p}{,}\PY{n}{tentativi}\PY{p}{,}\PY{l+s+s2}{\PYZdq{}}\PY{l+s+s2}{tentativi}\PY{l+s+s2}{\PYZdq{}}\PY{p}{)}
        \PY{k}{break}\PY{c+c1}{\PYZsh{}fa fermare \PYZdq{}improvvisamente\PYZdq{} il programma}
    \PY{k}{elif} \PY{n}{tentativo}\PY{o}{\PYZlt{}}\PY{n}{numerodaindovinare}\PY{p}{:}
        \PY{n+nb}{print}\PY{p}{(}\PY{l+s+s2}{\PYZdq{}}\PY{l+s+s2}{Ci sei quasi, ti stai avvicinando! Ti do un consiglio: il numero è più grande}\PY{l+s+s2}{\PYZdq{}}\PY{p}{)}
    \PY{k}{else}\PY{p}{:}
        \PY{n+nb}{print}\PY{p}{(}\PY{l+s+s2}{\PYZdq{}}\PY{l+s+s2}{Ci sei quasi, ti stai avvicinando! Ti do un consiglio: il numero è più piccolo}\PY{l+s+s2}{\PYZdq{}}\PY{p}{)}
\end{Verbatim}
\end{tcolorbox}

    \begin{Verbatim}[commandchars=\\\{\}]
Ciao, questo gioco prevede che tu indovini un numero compreso da 1 a 100
Adesso inserisci il numero che ritieni sia quello giusto: 10
Ci sei quasi, ti stai avvicinando! Ti do un consiglio: il numero è più grande
Adesso inserisci il numero che ritieni sia quello giusto: 90
Ci sei quasi, ti stai avvicinando! Ti do un consiglio: il numero è più piccolo
Adesso inserisci il numero che ritieni sia quello giusto: 20
Ci sei quasi, ti stai avvicinando! Ti do un consiglio: il numero è più grande
Adesso inserisci il numero che ritieni sia quello giusto: 80
Ci sei quasi, ti stai avvicinando! Ti do un consiglio: il numero è più piccolo
Adesso inserisci il numero che ritieni sia quello giusto: 30
Ci sei quasi, ti stai avvicinando! Ti do un consiglio: il numero è più grande
Adesso inserisci il numero che ritieni sia quello giusto: 70
Ci sei quasi, ti stai avvicinando! Ti do un consiglio: il numero è più piccolo
Adesso inserisci il numero che ritieni sia quello giusto: 40
Ci sei quasi, ti stai avvicinando! Ti do un consiglio: il numero è più grande
Adesso inserisci il numero che ritieni sia quello giusto: 60
Ci sei quasi, ti stai avvicinando! Ti do un consiglio: il numero è più grande
Adesso inserisci il numero che ritieni sia quello giusto: 61
Ci sei quasi, ti stai avvicinando! Ti do un consiglio: il numero è più grande
Adesso inserisci il numero che ritieni sia quello giusto: 62
Ci sei quasi, ti stai avvicinando! Ti do un consiglio: il numero è più grande
Adesso inserisci il numero che ritieni sia quello giusto: 63
Ci sei quasi, ti stai avvicinando! Ti do un consiglio: il numero è più grande
Adesso inserisci il numero che ritieni sia quello giusto: 64
Ci sei quasi, ti stai avvicinando! Ti do un consiglio: il numero è più grande
Adesso inserisci il numero che ritieni sia quello giusto: 65
Ci sei quasi, ti stai avvicinando! Ti do un consiglio: il numero è più grande
Adesso inserisci il numero che ritieni sia quello giusto: 67
Ci sei quasi, ti stai avvicinando! Ti do un consiglio: il numero è più grande
Adesso inserisci il numero che ritieni sia quello giusto: 68
Complimenti! Sei riuscito a indovinare il numero incognito che è 68 in 15
tentativi
    \end{Verbatim}

    Qui sotto è presente un breve codice che mostra come generare un numero
causale intero (int), questa parte di codice è anche presente nel codice
soprastante ma ovviamente con variabili diverse.

    \begin{tcolorbox}[breakable, size=fbox, boxrule=1pt, pad at break*=1mm,colback=cellbackground, colframe=cellborder]
\prompt{In}{incolor}{19}{\boxspacing}
\begin{Verbatim}[commandchars=\\\{\}]
\PY{c+c1}{\PYZsh{}Comprendere l\PYZsq{}utillizzo della libreria random per generare numeri casuali interi}
\PY{k+kn}{import} \PY{n+nn}{random}
\PY{n+nb}{print}\PY{p}{(}\PY{l+s+s2}{\PYZdq{}}\PY{l+s+s2}{Un numero di esempio intero (int) generato casualmente da un range da 1 a 100 è: }\PY{l+s+s2}{\PYZdq{}}\PY{p}{)}
\PY{n}{numerointerocasuale}\PY{o}{=}\PY{n}{random}\PY{o}{.}\PY{n}{randint}\PY{p}{(}\PY{l+m+mi}{1}\PY{p}{,}\PY{l+m+mi}{100}\PY{p}{)}\PY{c+c1}{\PYZsh{}random.randint vuol dire che il numero da generare è compreso da 1 a 100 ed è intero perchè è int}
\PY{n+nb}{print}\PY{p}{(}\PY{n}{numerointerocasuale}\PY{p}{)}
\end{Verbatim}
\end{tcolorbox}

    \begin{Verbatim}[commandchars=\\\{\}]
Un numero di esempio intero (int) generato casualmente da un range da 1 a 100 è:
27
    \end{Verbatim}

    Invece qui sotto è presente il metodo per farlo con i numeri decimali
(float).

    \begin{tcolorbox}[breakable, size=fbox, boxrule=1pt, pad at break*=1mm,colback=cellbackground, colframe=cellborder]
\prompt{In}{incolor}{22}{\boxspacing}
\begin{Verbatim}[commandchars=\\\{\}]
\PY{c+c1}{\PYZsh{}Comprendere l\PYZsq{}utillizzo della libreria random per generare numeri casuali decimali}
\PY{k+kn}{import} \PY{n+nn}{random}
\PY{n+nb}{print}\PY{p}{(}\PY{l+s+s2}{\PYZdq{}}\PY{l+s+s2}{Un numero di esempio decimale (float) generato casualmente da un range da 0,12 a 99,7 è: }\PY{l+s+s2}{\PYZdq{}}\PY{p}{)}
\PY{n}{numerofloatcasuale}\PY{o}{=}\PY{n}{random}\PY{o}{.}\PY{n}{uniform}\PY{p}{(}\PY{l+m+mf}{0.12}\PY{p}{,} \PY{l+m+mf}{99.7}\PY{p}{)}\PY{c+c1}{\PYZsh{}random.uniform vuol dire che il numero da generare è compreso da 0,12 a 99,07 ed è decimale perchè è float}
\PY{n+nb}{print}\PY{p}{(}\PY{n}{numerofloatcasuale}\PY{p}{)}
\end{Verbatim}
\end{tcolorbox}

    \begin{Verbatim}[commandchars=\\\{\}]
Un numero di esempio decimale (float) generato casualmente da un range da 0,12 a
99,7 è:
86.19652177433156
    \end{Verbatim}

    \section{IL GIOCO DEL MORRA CINESE}\label{il-gioco-del-morra-cinese}

    Nel codice del gioco della morra cinese, viene sfruttata la libreria
random di Python per creare un gioco interattivo e coinvolgente.
Inizialmente, il programma genera casualmente la scelta da parte del
computer tra carta, forbici e sasso, consentendo così una migliore
dinamicità e casualità nel gioco. Successivamente, l'utente viene
coinvolto con un'interfaccia di input, in cui è richiesto di selezionare
(cioè scrivere nella apposita casella di testo) la propria mossa
preferita tra le opzioni disponibili: carta, forbici o sasso. Il
programma fornisce poi all'utente la mossa scelta dal computer
stampandola a schermo appena l'utente avrà scritto la sua mossa, creando
così un'atmosfera di suspense. Le regole del gioco sono implementate
attraverso delle condizioni logiche che confrontano le mosse dell'utente
e del computer con delle scelte prefissate. Gli operatori logici and e
or vengono utilizzati per definire le situazioni in cui l'utente può
vincere, perdere o ottenere una parità contro il computer e vengono
usate per poter specificare più cose contemporaneamente senza dover
creare tantissime condizioni nel codice. Inoltre, il codice gestisce
situazioni in cui l'utente inserisce una mossa non riconosciuta rispetto
all'elenco delle opzioni scritte sopra, fornendo un messaggio di avviso
e suggerendo di scegliere tra le opzioni valide riavviando però prima il
programma. Un elemento interessante del codice è l'inclusione di un link
esterno, in questo caso un link di Wikipedia, che invita l'utente a
consultare ulteriori informazioni sul regolamento e le strategie del
gioco della morra cinese. Complessivamente, il codice combina
efficacemente l'input utente, logica condizionale e interazione dinamica
per creare un'esperienza di gioco coinvolgente. Questo codice riprende
quasi tutte le competenze raccolte finora in questa esercitazione.

    \begin{tcolorbox}[breakable, size=fbox, boxrule=1pt, pad at break*=1mm,colback=cellbackground, colframe=cellborder]
\prompt{In}{incolor}{27}{\boxspacing}
\begin{Verbatim}[commandchars=\\\{\}]
\PY{c+c1}{\PYZsh{}Riassunto di tutte le competenze imparate finora in questa esercitazione + comprendere l\PYZsq{}utilizzo dei comandi not in, and e or}
\PY{k+kn}{import} \PY{n+nn}{random}
\PY{n}{mosse}\PY{o}{=}\PY{p}{[}\PY{l+s+s2}{\PYZdq{}}\PY{l+s+s2}{carta}\PY{l+s+s2}{\PYZdq{}}\PY{p}{,}\PY{l+s+s2}{\PYZdq{}}\PY{l+s+s2}{forbici}\PY{l+s+s2}{\PYZdq{}}\PY{p}{,}\PY{l+s+s2}{\PYZdq{}}\PY{l+s+s2}{sasso}\PY{l+s+s2}{\PYZdq{}}\PY{p}{]}
\PY{n}{mossadelcomputer}\PY{o}{=}\PY{n}{random}\PY{o}{.}\PY{n}{choice}\PY{p}{(}\PY{n}{mosse}\PY{p}{)}\PY{c+c1}{\PYZsh{}random.choice vuol dire che il programma sceglierà casualmente un elemento tra quelli della lista o stringa}
\PY{n+nb}{print}\PY{p}{(}\PY{l+s+s2}{\PYZdq{}}\PY{l+s+s2}{Ciao, questo gioco è la morra cinese quindi devi provare ad indovinare la mossa giusta e battere il computer che è l}\PY{l+s+s2}{\PYZsq{}}\PY{l+s+s2}{avversario}\PY{l+s+s2}{\PYZdq{}}\PY{p}{)}
\PY{n+nb}{print}\PY{p}{(}\PY{l+s+s2}{\PYZdq{}}\PY{l+s+s2}{Puoi trovare tutto il regolamento e le strategie del gioco a questo link: https://it.wikipedia.org/wiki/Morra\PYZus{}cinese}\PY{l+s+s2}{\PYZdq{}}\PY{p}{)}
\PY{n}{sceltadelgiocatore}\PY{o}{=}\PY{n+nb}{input}\PY{p}{(}\PY{l+s+s2}{\PYZdq{}}\PY{l+s+s2}{Quindi adesso scegli la tua mossa scrivendo una delle seguenti opzioni: carta, forbice o sasso: }\PY{l+s+s2}{\PYZdq{}}\PY{p}{)}
\PY{k}{if} \PY{n}{sceltadelgiocatore} \PY{o+ow}{not} \PY{o+ow}{in} \PY{n}{mosse}\PY{p}{:}\PY{c+c1}{\PYZsh{}il comando not in viene utilizzato per verificare se un elemento NON è presente in una lista o stringa}
    \PY{n+nb}{print}\PY{p}{(}\PY{l+s+s2}{\PYZdq{}}\PY{l+s+s2}{Mossa non riconosciuta. Scegli una delle seguenti mosse riavviando prima il programma: carta, forbice o sasso}\PY{l+s+s2}{\PYZdq{}}\PY{p}{)}
    \PY{n+nb}{print}\PY{p}{(}\PY{l+s+s2}{\PYZdq{}}\PY{l+s+s2}{Impossibile perciò decretare un vincitore}\PY{l+s+s2}{\PYZdq{}}\PY{p}{)}
\PY{k}{else}\PY{p}{:}\PY{c+c1}{\PYZsh{}all\PYZsq{}interno di un else, o in una qualsiasi altra condizione, ci possono essere altre condizioni che si possono verificare, con altre condizioni a loro volta all\PYZsq{}interno e così via dicendo}
    \PY{n+nb}{print}\PY{p}{(}\PY{l+s+s2}{\PYZdq{}}\PY{l+s+s2}{Il computer ha scelto la seguente mossa:}\PY{l+s+s2}{\PYZdq{}}\PY{p}{,}\PY{n}{mossadelcomputer}\PY{p}{)}
    \PY{k}{if} \PY{n}{sceltadelgiocatore}\PY{o}{==}\PY{n}{mossadelcomputer}\PY{p}{:}
        \PY{n+nb}{print}\PY{p}{(}\PY{l+s+s2}{\PYZdq{}}\PY{l+s+s2}{Parità!}\PY{l+s+s2}{\PYZdq{}}\PY{p}{)}
    \PY{k}{elif} \PY{p}{(}\PY{n}{sceltadelgiocatore}\PY{o}{==}\PY{l+s+s2}{\PYZdq{}}\PY{l+s+s2}{carta}\PY{l+s+s2}{\PYZdq{}} \PY{o+ow}{and} \PY{n}{mossadelcomputer}\PY{o}{==}\PY{l+s+s2}{\PYZdq{}}\PY{l+s+s2}{sasso}\PY{l+s+s2}{\PYZdq{}}\PY{p}{)}\PY{o+ow}{or}\PY{p}{(}\PY{n}{sceltadelgiocatore}\PY{o}{==}\PY{l+s+s2}{\PYZdq{}}\PY{l+s+s2}{forbici}\PY{l+s+s2}{\PYZdq{}} \PY{o+ow}{and} \PY{n}{mossadelcomputer}\PY{o}{==}\PY{l+s+s2}{\PYZdq{}}\PY{l+s+s2}{carta}\PY{l+s+s2}{\PYZdq{}}\PY{p}{)}\PY{o+ow}{or}\PY{p}{(}\PY{n}{sceltadelgiocatore}\PY{o}{==}\PY{l+s+s2}{\PYZdq{}}\PY{l+s+s2}{sasso}\PY{l+s+s2}{\PYZdq{}} \PY{o+ow}{and} \PY{n}{mossadelcomputer}\PY{o}{==}\PY{l+s+s2}{\PYZdq{}}\PY{l+s+s2}{forbici}\PY{l+s+s2}{\PYZdq{}}\PY{p}{)}\PY{p}{:}\PY{c+c1}{\PYZsh{}l\PYZsq{}and serve a comparare più cose contemporaneamente mentre l\PYZsq{}or funge come un oppure}
        \PY{n+nb}{print}\PY{p}{(}\PY{l+s+s2}{\PYZdq{}}\PY{l+s+s2}{Complimenti! Hai vinto!}\PY{l+s+s2}{\PYZdq{}}\PY{p}{)}
    \PY{k}{else}\PY{p}{:}
        \PY{n+nb}{print}\PY{p}{(}\PY{l+s+s2}{\PYZdq{}}\PY{l+s+s2}{Peccato! Hai perso!}\PY{l+s+s2}{\PYZdq{}}\PY{p}{)}
\end{Verbatim}
\end{tcolorbox}

    \begin{Verbatim}[commandchars=\\\{\}]
Ciao, questo gioco è la morra cinese quindi devi provare ad indovinare la mossa
giusta e battere il computer che è l'avversario
Puoi trovare tutto il regolamento e le strategie del gioco a questo link:
https://it.wikipedia.org/wiki/Morra\_cinese
Quindi adesso scegli la tua mossa scrivendo una delle seguenti opzioni: carta,
forbice o sasso: forbici
Il computer ha scelto la seguente mossa: carta
Complimenti! Hai vinto!
    \end{Verbatim}


    % Add a bibliography block to the postdoc
    
    
    
\end{document}
